\documentclass[11pt]{article}
\usepackage[utf8]{inputenc}
\usepackage[spanish, mexico]{babel}
\usepackage{listings}
\usepackage{breakcites}
\usepackage{dsfont}
\usepackage{hyperref}
\usepackage{amssymb,amsthm,amsmath,latexsym}
\usepackage[margin=1.5cm]{geometry}
\usepackage{natbib}
%\bibliographystyle{stylename}
%\usepackage{fancyhdr}
%\pagestyle{fancy}
\theoremstyle{plain}
\newtheorem{teo}{Teorema}
\newtheorem{prop}[teo]{Proposición}
\newtheorem{defi}[teo]{Definición}
\newtheorem{obs}[teo]{Observación}
\newtheorem{lem}[teo]{Lema}
\newtheorem{cor}[teo]{Corolario}
\usepackage[pdftex]{color,graphicx}
\usepackage{tikz}
\usetikzlibrary{calc}
%\newcommand{\sgn}{\mathop{\mathrm{sgn}}}
\title{Adquisición y uso de relaciones causales para la toma de decisiones bajo incertidumbre.}
\author{Mauricio Gonzalez Soto}
\begin{document}
%\nocite{*}
\maketitle
\section{Introducción}

\section{Marco Teórico}
\subsection{Relaciones Causales}
\subsection{Toma de decisiones}
\subsection{Teoría de Juegos}
\subsection{Trabajo Previo}
\section{Objetivos}

\section{Metodología}

\section{Resultados Preliminares}
\subsection{Caso en el que se conoce el modelo completo}
\subsection{Caso en el que se conoce la estructura del modelo}
\subsection{Caso general}
\section{Conclusiones}

\bibliographystyle{apalike}
\bibliography{/Users/MauricioGS1/INAOE/Segundo_Semestre/Propuesta/Bibliografia.bib}
\end{document}