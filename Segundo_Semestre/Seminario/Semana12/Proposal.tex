\documentclass[11pt]{article}
\usepackage[utf8]{inputenc}
\usepackage[spanish, mexico]{babel}
\usepackage{listings}
\usepackage{breakcites}
\usepackage{dsfont}
\usepackage{hyperref}
\usepackage{amssymb,amsthm,amsmath,latexsym}
\usepackage[margin=1.5cm]{geometry}
\usepackage{natbib}
%\bibliographystyle{stylename}
%\usepackage{fancyhdr}
%\pagestyle{fancy}
\theoremstyle{plain}
\newtheorem{teo}{Teorema}
\newtheorem{prop}[teo]{Proposición}
\newtheorem{defi}[teo]{Definición}
\newtheorem{obs}[teo]{Observación}
\newtheorem{lem}[teo]{Lema}
\newtheorem{cor}[teo]{Corolario}
\usepackage[pdftex]{color,graphicx}
\usepackage{tikz}
\usetikzlibrary{calc}
%\newcommand{\sgn}{\mathop{\mathrm{sgn}}}
\title{Adquisición y uso de relaciones causales para la toma de decisiones bajo incertidumbre.}
\author{Mauricio Gonzalez Soto}
\begin{document}
%\nocite{*}
\maketitle
\section{Introducción}

\section{Marco Teórico}
	\subsection{Relaciones Causales}
		\subsubsection{Definición de Causalidad}
		La definición de Causalidad que estamos considerando es la que aparece en \cite{spirtes2000causation} y que consiste en lo siguiente:
		\begin{defi}
		Dado un espacio de probabilidad $(\Omega, \mathcal{F},\mathbb{P})$, consideramos una relación binaria $\to$ definida sobre elementos de $\mathcal{F}$ que sea
		\begin{itemize}
		\item Transitiva
		\item Irefflexiva
		\item Antisimétrica
		\end{itemize}
		\end{defi}
		De esta manera, si $A \to B$ decimos que $A$ es una \textit{causa directa} de $B$ y que $B$ es el efecto de la ocurrencia de $A$. Abusando de la notación, podemos construir un grafo que 	         represente las relaciones  causales entre eventos de $\mathcal{F}$ de la siguiente manera: si $A \to B$, entonces se tiene un nodo que representa el evento $A$ y un nodo que representa el evento $B$ y una arista dirigida que los une en el sentido de la causa hacia el efecto. Además, requerimos las siguientes condiciones que relacionan el grafo así construido con la distribución de probabilidad que éste expresa:
		\begin{itemize}
		\item Markov Causal
		\item Minimalidad causal
		\item Fidelidad Causal
		\end{itemize}
		\subsubsection{Modelos gráficos causales}
		Un modelo gráfico causal consiste en un conjunto de variables aleatorias $\mathcal{X}=\{ X_1,...,X_n \}$, un grafo $\mathcal{G}$ cuyos nodos corresponden a variables en $\mathcal{X}$ y las aristas entre ellos a relaciones causales. Además, un operador $do()$ que está definido sobre grafos y cuya acción corresponde a lo siguiente: Dado $\mathbf{X} \subseteq \mathcal{X}$ y $\mathbf{x} = \{ x_{i_1}, x_{i_2}, ... , x_{i_j} \} \in Val(\mathcal{X})$, $do(\mathbf{X} = \mathbf{x} )$ consiste en asignar a cada $X_j \in \mathbf{X}$ el valor $x_{x_{i_j}}$ y eliminar todas las aristas que entran al nodo correspondiente a $X_i$ en el grafo $\mathcal{G}$.
		\subsubsection{El problema de la identificabilidad}
		\subsubsection{Cálculo Do}
	\subsection{Toma de decisiones}
	\subsection{Teoría de Juegos}
	\subsection{Trabajo Previo}

\section{Objetivos}
	\section{Objetivos Generales}
	\section{Objetivos específicos}
	
\section{Metodología}
	\subsection{Hipótesis}
	\subsection{Metodología de Trabajo}

\section{Resultados Preliminares}
Para mostrar la factibilidad del modelo planteado, consideramos por separado tres casos cuyo grado de dificultad es ascendente. En el primer caso, supondremos que el modelo causal es conocido completamente por el agente y veremos cuál sería su manera de actuar en este caso. Que el modelo sea conocido completamente significa que el agente conoce la estructura del grafo así como las probabilidades asociadas a cada variable. Posteriormente, en el segundo caso, el agente sólo tendrá a su disposición la estructura del grafo, pero no los parámetros de este. En el tercer caso, el agente no conocerá nada del modelo, pero tendrá \textit{creencias} sobre este; este caso generaliza los anteriores, pues en el primer y segundo caso las creencias con las que inicia el agente es la información que conoce sobre el modelo
\subsection{Caso en el que se conoce el modelo completo}
\subsection{Caso en el que se conoce la estructura del modelo}
\subsection{Caso general}
\section{Conclusiones}

\bibliographystyle{apalike}
\bibliography{/Users/MauricioGS1/INAOE/Segundo_Semestre/Propuesta/Bibliografia.bib}
\end{document}