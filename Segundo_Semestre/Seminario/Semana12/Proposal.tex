\documentclass[11pt]{article}
\usepackage[utf8]{inputenc}
\usepackage[spanish, mexico]{babel}
\usepackage{listings}
\usepackage{breakcites}
\usepackage{dsfont}
\usepackage{hyperref}
\usepackage{amssymb,amsthm,amsmath,latexsym}
\usepackage[margin=1.5cm]{geometry}
\usepackage{natbib}
%\bibliographystyle{stylename}
%\usepackage{fancyhdr}
%\pagestyle{fancy}
\theoremstyle{plain}
\newtheorem{teo}{Teorema}
\newtheorem{prop}[teo]{Proposición}
\newtheorem{defi}[teo]{Definición}
\newtheorem{obs}[teo]{Observación}
\newtheorem{lem}[teo]{Lema}
\newtheorem{cor}[teo]{Corolario}
\usepackage[pdftex]{color,graphicx}
\usepackage{tikz}
\usetikzlibrary{calc}
%\newcommand{\sgn}{\mathop{\mathrm{sgn}}}
\title{Adquisición y uso de relaciones causales para la toma de decisiones bajo incertidumbre.}
\author{Mauricio Gonzalez Soto}
\begin{document}
%\nocite{*}
\maketitle
\section{Introducción}

\section{Marco Teórico}
	\subsection{Relaciones Causales}
		\subsubsection{Definición de Causalidad}
		\subsubsection{Modelos gráficos causales}
		\subsubsection{El problema de la identificabilidad}
		\subsubsection{Cálculo Do}
	\subsection{Toma de decisiones}
	\subsection{Teoría de Juegos}
	\subsection{Trabajo Previo}

\section{Objetivos}
	\section{Objetivos Generales}
	\section{Objetivos específicos}
	
\section{Metodología}
	\subsection{Hipótesis}
	\subsection{Metodología de Trabajo}

\section{Resultados Preliminares}
Para mostrar la factibilidad del modelo planteado, consideramos por separado tres casos cuyo grado de dificultad es ascendente. En el primer caso, supondremos que el modelo causal es conocido completamente por el agente y veremos cuál sería su manera de actuar en este caso. Que el modelo sea conocido completamente significa que el agente conoce la estructura del grafo así como las probabilidades asociadas a cada variable. Posteriormente, en el segundo caso, el agente sólo tendrá a su disposición la estructura del grafo, pero no los parámetros de este. En el tercer caso, el agente no conocerá nada del modelo, pero tendrá \textit{creencias} sobre este; este caso generaliza los anteriores, pues en el primer y segundo caso las creencias con las que inicia el agente es la información que conoce sobre el modelo
\subsection{Caso en el que se conoce el modelo completo}
\subsection{Caso en el que se conoce la estructura del modelo}
\subsection{Caso general}
\section{Conclusiones}

\bibliographystyle{apalike}
\bibliography{/Users/MauricioGS1/INAOE/Segundo_Semestre/Propuesta/Bibliografia.bib}
\end{document}