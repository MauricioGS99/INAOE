\documentclass[11pt]{article}
\usepackage[utf8]{inputenc}
\usepackage[spanish, mexico]{babel}
\usepackage{listings}
\usepackage{breakcites}
\usepackage{dsfont}
\usepackage{hyperref}
\usepackage{amssymb,amsthm,amsmath,latexsym}
\usepackage[margin=1.5cm]{geometry}
\usepackage{natbib}
%\bibliographystyle{stylename}
%\usepackage{fancyhdr}
%\pagestyle{fancy}
\theoremstyle{plain}
\newtheorem{teo}{Teorema}
\newtheorem{prop}[teo]{Proposición}
\newtheorem{defi}[teo]{Definición}
\newtheorem{obs}[teo]{Observación}
\newtheorem{lem}[teo]{Lema}
\newtheorem{cor}[teo]{Corolario}
\usepackage[pdftex]{color,graphicx}
\usepackage{tikz}
\usetikzlibrary{trees}
\usetikzlibrary{calc}
%\newcommand{\sgn}{\mathop{\mathrm{sgn}}}
\title{Adquisición y uso de relaciones causales para la toma de decisiones bajo incertidumbre utilizando juegos con información imperfecta.}
\author{Mauricio Gonzalez Soto}
\begin{document}
%\nocite{*}
\maketitle
\section{Introducción}
Debido al enorme número de factores diversos que están involucrados en nuestro entorno, constántemente nos vemos forzados a tomar decisiones cuyas consecuencias son inciertas (\cite{danks2014unifying}). Esa misma realidad, altamente compleja, contiene una serie de relaciones causa-efecto que podemos utilizar al momento de tomar decisiones. El conocimiento causal permite preguntarnos  \textit{qué pasaría si...} (\cite{stalnaker2016knowledge}) y predecir las consecuencias de nuestros actos. Por desgracia, es difícil que contemos con la información necesaria para llevar a cabo estas predicciones de manera correcta, a lo más, podemos contar con estimaciones sobre lo probable que es la ocurrencia de algún evento incierto dada cierta acción.\\
\\
\indent La Teoría de la Decisión bajo Incertidumbre, desarrollada en los trabajos de \cite{von1944theory}, \cite{definetti1930}, \cite{definetti1937}, \cite{savage1954the}, \cite{bernardo2000bayesian}, considera a un tomador de decisiones que se enfrenta al problema de escoger una de entre varias opciones (o acciones) posibles en un escenario incierto, el cual influirá en las consecuencias de las acciones. Para resolver este problema a partir de una serie de axiomas se propone un criterio formal de elección, el cual consiste en la maximización de la utilidad esperada.\\
\\
\indent La utilidad esperada, como su nombre lo indica es el valor esperado, o promedio, de la utilidad que le producen al tomador de decisiones las consecuencias de sus acciones; para poder calcularla, es necesario contar con estimaciones de la probabilidad de ocurrencia de ciertos eventos inciertos. El Teorema de Von Neuman-Morgenstern, garantiza la existencia de una función de utilidad si un tomador de decisiones conoce las probabilidades de los eventos, entonces  la relación de preferencias del agente es equivalente al orden que impone la utilidad esperada respecto a esta función y las probabilidades conocidas, por lo que la mejor decisión a tomar es aquella que maximice esta cantidad.\\

\indent Por otro lado, el Teorema de DeFinnetti supone conocida la utilidad para el agente y el resultado garantiza la existencia de una medida de probabilidad (subjetiva) tal que la relación de preferencias es equivalente a maximizar la utilidad respecto a esta medida. De esta manera, se resuelve parcialmente el problema de toma de decisiones, pues si por un lado se cuenta con las probabilidades de eventos inciertos el Teorema de von Neumann devuelve la utilidad, y si se cuenta con la utilidad para el tomador de decisiones, DeFinetti devuelve las probabiliddes, pero ‚por dónde empezar?  
\\
\indent El gran logro de J.L. Savage fue derivar  la utilidad y la probabilidad en conjunto en vez de utilizarlas como primitivas del modelo (\cite{gilboa2009decision}). Lo único que requiere el Teorema de Savage es una serie de axiomas de coherencia sobre la relación de preferencias de un tomador de decisiones; con esto, muestra la existencia de una función de utilidad y de una medida de probabilidad tal que la relación de preferencias es equivalente a la maximización de la utilidad esperada. Existen otros criterios de optimalidad cuando los axiomas de Savage son cuestionados; por ejemplo, la utilidad maxmin, o la utilidad esperada de Choquet (\cite{gilboa2009decision}). \cite{gilboa2001theory} propone un nuevo paradigma de toma de decisiones en el cual los agentes toman decisiones haciendo analogías a eventos pasados en los cuales tuvieron un buen desempeño. Estas teorías, aunque interesantes, no serán abordadas en este trabajo sino que se mencionan sólo por completez.\\
\\
\indent En diversas ocasiones, un tomador de decisiones no tiene a su alcance todos los parámetros (probabilidades de los eventos), pero puede intentar \textit{aprenderlos} del mismo entorno, sobre esto ya se ha dicho mucho en el área de Aprendizaje por Refuerzo (\cite{sutton1998reinforcement}). Una de las grandes limitaciones de estos métodos es que aprenden formas de actuar que son puramente reactivas; es decir, no capturan, ni intentan, la estructura del ambiente. Un caso particular de esta estructura se da en el caso en el cual el tomador de decisiones sabe, o cree, que en el ambiente en el cual se llevan a cabo sus decisiones existen relaciones causa-efecto.
	\subsection{Motivación}
	Las relaciones causa-efecto\footnote{Entendiendo causalidad como la define \cite{spirtes2000causation}; es decir, como una relación transitiva, irreflexiva y antisimétrica entre eventos, de manera que sucede uno y esto causa que otro evento sea producido.} son de gran utilidad en problemas de decisión, pues conociendo los \textit{efectos} de las posibles decisiones a tomar, el agente puede planear mejor sus acciones (\cite{hagmayer2013repeated},\cite{pearlwhy}). Además, las relaciones causales son un tipo de \textit{estructura} del ambiente, la cual puede utilizarse para que un agente pueda razonar sobre su entorno. De hecho, se ha estudiado el caso de agentes que son racionales en un entorno causal (\cite{board2006equivalence}). Además, en el área de la Economía conocida como Economía Conductual (Behavioral Economics) es un problema interesante el dotar agentes que tienen un comportamiento incoherente en el tiempo (\cite{kleinberg2014time}) con incentivos, lo cual se puede hacer si se conocen una estructura de dependencia entre acciones y recompensas (\cite{albers2016motivating}). \\
\\
\indent La información causal por sí sola nos ayuda a los seres humanos a tomar decisiones, pues resulta que los seres humanos utilizan información causal al tomar decisiones; además, modifican la información causal previa que tenían del ambiente conforme interactúan con este (\cite{hagmayer2013repeated}). Además, este conocimiento causal es utilizado de manera correcta por los seres humanos para escoger las acciones que lleven a cierto resultado deseado (\cite{sloman2006causal}, \cite{nichols2007decision}, \cite{meder2010observing}, \cite{hagmayer2013repeated}, \cite{danks2014unifying}). Aunque, los seres humanos no son perfectos al estimar la fuerza de los efectos, o violan el supuesto Markoviano (\cite{rottman2014reasoning}). \\
\\
\indent De la misma manera en la que a los seres humanos nos ayuda contar con información causal, \cite{lattimoreNIPS2016} muestran que un agente, un tomador de decisiones, que cuenta con un modelo causal puede encontrar de manera más rápida la acción óptima en un contexto incierto. En su trabajo muestran, empírica y teóricamente, que contar con un modelo causal lleva a una mejor toma de decisiones que si sólo se realizaran búsquedas asociativas.\\
\\
\indent Han existido diversos intentos de formalizar una Teoría de Decisión que incorpore información causal, como han sido los trabajos de \cite{joyce1999foundations}, \cite{board2006equivalence}, \cite{joyce2012regret}, \cite{ahmed2012push}, \cite{rottman2014reasoning}, \cite{soares2015toward}, \cite{stalnaker2016knowledge}, los cuales hacen uso del “razonamiento contrafactual” para proponer una manera de tomar decisiones a la luz de “lo que hubiera pasado si...”. Estos trabajos tienen ciertas limitaciones filosóficas y conceptuales, pues ignoran el caso en el cual las acciones de un tomador de decisiones están correlacionadas (pero no causalmente conectadas) con algunas partes del ambiente, además de la dificultad que implica la construcción de un modelo causal. \\
\\
\indent Teoría de Juegos (\cite{osborne1994course}) es un área de las matemáticas que los economistas y otros científicos sociales utilizan para modelar situaciones de interacción estratégica; es decir, problemas de decisión en los cuales importa no solo las acciones y preferencias de un sólo agente sino las de otros. En particular, en esta área existen los llamados \textit{juegos Bayesianos}, también conocidos como Juegos Extensivos con Información Incompleta (\cite{osborne1994course}, \cite{10.1007/978-94-010-0189-2_25}),  en los cuales cada jugador tiene creencias \textit{a priori} sobre algunos aspectos del ambiente (en las aplicaciones de economía estas creencias suelen versar sobre las recompensas o sobre las estrategias de otros jugadores). En general, los juegos Bayesianos pertenecen a una clase de juegos llamados juegos estratégicos.\\
\\
Las creencias que suele tener un jugador suelen ser sobre ciertos aspectos; por ejemplo, creencias sobre las acciones de los otros jugadores (\cite{costa2008stated}), creencias sobre el estado del mundo (\cite{dominitz2009empirical}) y creencias sobre las creencias de los otros jugadores
\\
\cite{tingley2010belief}
\subsection{Justificación}
Contar con información sobre la estructura causal de cierto entorno puede hacer que un proceso de toma de decisiones se lleve a cabo de manera eficiente; en el trabajo de \cite{lattimoreNIPS2016} se muestra cómo al contar con un modelo causal del ambiente se pueden encontrar acciones óptimas de manera más rápida que si sólo se llevara a cabo una exploración a ciegas con tal de encontrar la mejor acción como llevan a cabo \cite{audibert2010best}. \cite{lattimoreNIPS2016} consideran el caso en el cual un tomador de decisiones quiere maximizar cierta variable de recompensa y las acciones que tiene a su disposición el agente están relacionadas de manera causal con la variable recompensa y entre sí.\\
\\
\indent Se sabe que los seres humanos conciben sus acciones en el mundo como \textit{intervenciones} sobre este (\cite{hagmayer2009decision}). Siguiendo esta idea,  \cite{lattimoreNIPS2016} consideran las acciones disponibles a un agente inteligente como \textit{intervenciones} posibles en un modelo causal; la dinámica que proponen consiste en que el agente realice una acción y posterior a esto se observa el valor de la variable recompensa así como la realización de las otras variables del modelo que no fueron intervenidas; en esto consiste una \textit{ronda}. En la siguiente ronda, el entorno vuelve a su estado inicial y el agente selecciona otra posible intervención y vuelve a observar estas variables. Esta dinámica de rondas repetidas en la que cada ronda inicia sin tomar en cuenta lo sucedido en la anterior lleva a los autores a modelar el problema como escoger brazos en una máquina tragamonedas (conocidas como \textit{bandits} en la literatura), observar el pago y moverse a una máquina nueva. El algoritmo que desarrollan identifica la \textit{intervención óptima} después de un número $T$ de rondas fijado por el agente y permite hacerlo de manera más rápida que los algoritmos del estado del arte.\\
\\
\indent Lo que este trabajo señala es que  la luz de un modelo causal se pueden tomar decisiones que maximicen cierta variable de pago de manera más rápida que los métodos que no incorporan relaciones causales. La limitación que tiene este trabajo es suponer conocido de antemano el modelo causal, pero para modelar un escenario más realista es necesario suponer no conocido el modelo causal e intentar aprender sobre éste a partir de las interacciones mismas con el ambiente; de hecho es, en parte, como operamos los seres humanos (\cite{hagmayer2013repeated}). La importancia del razonamiento causal para el desarrollo de agentes inteligentes está detallada y justificada en \cite{lake2017building}.\\
\\
\indent Modificar uno a uno los vértices, o por decirlo de otra forma, hacer actualizaciones locales, como se hace en \cite{lattimoreNIPS2016} tiene sentido pues existe evidencia de que los seres humanos se enfocan en aprender aspectos locales de estructuras causales (\cite{danks2014unifying}) y posteriormente unificar estos aspectos en una sola estructura coherente (\cite{fernbach2009causal}, \cite{waldmann2008causal}, \cite{wellen2012learning}). Además, tomar en cuenta la restricción de sólo tener una decisión por ronda es realista, pues problemas de decisión secuencial pueden modelarse como sucesiones de decisiones de una sola acción, además de que siempre es posible considerar varias acciones simultáneas como una sola que las englobe.\\
\\
\indent A diferencia de lo que sucede en Aprendizaje por Refuerzo (RL por sus siglas en inglés), en donde el agente lleva a cabo una acción y observa una señal de recompensa del ambiente, lo que se propone es a observar una \textit{jugada} de la Naturaleza; es decir, toda una acción y no sólo una cuantificación de lo devuelto por el entorno. Diseñar una señal de recompensa es un problema difícil en general (\cite{sutton1998reinforcement},\cite{dewey2014reinforcement}, \cite{DRLnotwork}), pues es la manera de especificarle al agente qué debe hacer, pero no cómo. Además, pudiera ocurrir que la señal de recompensa sea construída de modo que oculte las relaciones causales al agente, por eso se opta por observar lo que ocurre afuera como jugadas y que sea el agente quien asigne una utilidad a los resultados finales del juego.
\section{Marco Teórico}
En esta sección se mencionarán las definiciones más importantes relativas a esta propuesta de investigación doctoral sin pretender ser exhaustivo. El contenido de estas secciones es para dar al lector certeza sobre la existencia y validez de los resultados que se pretenden utilizar. 
	\subsection{Relaciones Causales}
		\subsubsection{Definición de Causalidad}
		La definición de Causalidad que estamos considerando, que está inspirada en la definición que da  \cite{spirtes2000causation} consiste en lo siguiente:
		\begin{defi}
		Dado un espacio medible $(\Omega, \mathcal{F})$, consideramos una relación binaria $\to$ definida sobre elementos de $\mathcal{F}$ que sea
		\begin{itemize}
		\item Transitiva: si $A \to B$ y $B \to C$ para $A,B,C \in \mathcal{F}$ entonces $A \to C$.
		\item Irefflexiva: Para todo $A \in \mathcal{F}$ no es cierto que $A \to A$.
		\item Antisimétrica: Para $A, B \in \mathcal{F}$, si $A \to B$, entonces no se cumple que $B \to A$.
		\end{itemize}
		\end{defi}
		\subsubsection{Causas Directas e Indirectas}
		En términos de la definición anterior, dado $(\Omega, \mathcal{F})$ si se tiene que $A \to B$ para $A,B \in \mathcal{F}$, entonces decimos que $A$ es una \textit{causa} de $B$ y que $B$ es el efecto de la ocurrencia de $A$. Consideremos ahora un \textit{refinamiento} $\mathcal{F}_s$ de la sigma-álgebra $\mathcal{F}$ de modo que ahora puedan existir nuevos eventos $C_s \in \mathcal{F}_s$ tales que $A \to C_s \to B$. Decimos que $A$ es una \textit{causa directa} de $B$ si se cumple que $A \to B$ y no existe evento $C$ tal que $A \to C \to B$ para todo refinamiento propio de $\mathcal{F}$. Notemos que la definición depende del conjunto de variables, por lo que al decir que un evento es causa directa de otro, es siempre relativo al conjunto de variables. 
		
		\subsubsection{Construcción de un grafo que represente relaciones causales}
		Abusando de la notación, podemos construir un grafo que represente las relaciones causales entre eventos de $\mathcal{F}$ de la siguiente manera: si $A \to B$ y $A$ es una causa directa de $B$, entonces se añade un nodo que representa el evento $A$ y un nodo que representa el evento $B$ y una arista dirigida que los une en el sentido de la causa hacia el efecto. A partir de definición anterior de relación causal se obtiene un grafo acíclico dirigido.\\
\\		
		 Dado un grafo acíclico dirigido, podemos construir una medida de probabilidad que expresa las relaciones que se encuentran establecidas en el grafo, la cual denotaremos $P_{\mathcal{G}}$. Requerimos que se cumplan las siguientes condiciones que relacionan el grafo así construido con la distribución de probabilidad $P_{\mathcal{G}}$ que éste expresa:
		\begin{itemize}
		\item Markov Causal
		\item Minimalidad Causal
		\item Fidelidad Causal
		\end{itemize}
		\subsubsection{Modelos gráficos causales}
		Un modelo gráfico causal consiste en un conjunto de variables aleatorias $\mathcal{X}=\{ X_1,...,X_n \}$, un grafo acíclico dirigido $\mathcal{G}$ cuyos nodos corresponden a variables en $\mathcal{X}$ y las aristas entre ellos a causas directas. Las variables de $\mathcal{X}$ están divididas en dos subconjuntos: las variables endógenas y las variables exógenas, en donde estas últimas son aquellas que no tienen padres en el grafo, esto se deriva naturalmente de la construcción del grafo a partir de las relaciones causales. Además, se tiene un operador $do()$ que está definido sobre grafos y cuya acción corresponde a lo siguiente: Dado $\mathbf{X} \subseteq \mathcal{X}$ y $\mathbf{x} = \{ x_{i_1}, x_{i_2}, ... , x_{i_j} \} \in Val(\mathcal{X})$, $do(\mathbf{X} = \mathbf{x} )$ consiste en asignar a cada $X_j \in \mathbf{X}$ el valor $x_{x_{i_j}}$ y eliminar todas las aristas que entran al nodo correspondiente a $X_i$ en el grafo $\mathcal{G}$. Más adelante entraremos en detalle respecto a este operador.
		\subsubsection{El problema de la identificabilidad} 
		
		\subsubsection{Cálculo Do}
		El cálculo do (\cite{pearl1995causal}, \cite{pearl2009causality}) consiste en un conjunto de reglas de inferencia que permiten manipular enunciados sobre observaciones que provienen de 		intervenciones. Consideremos un modelo gráfico causal $\mathcal{G}$ y sean $X,Y,Z$ conjuntos disjuntos de nodos de $\mathcal{G}$. Como notación, $\mathcal{G}_{\bar{X}}$ será el grafo que 	se obtiene al eliminar de $\mathcal{G}$ todas las aristas que entran en elementos de $X$; análogamente, $\mathcal{G}_{\underline{X}}$ será el grafo que se obtiene al eliminar de $\mathcal{G}$ todas las aristas que salen de elementos de $X$ y por último, $\mathcal{G}_{\underline{Z}\bar{X}}$ al eliminar arcos que entran de uno y que salen de otro conjunto.\\
	\\
		Las reglas de inferencia consisten en lo siguiente:
		\begin{teo}{\label{docalculus}} (\cite{pearl2009causality})\\
		Sea $\mathcal{G}$ un modelo gráfico causal y $P_{\mathcal{G}}$ la medida de probabilidad inducida por el modelo; entonces, para cualesquiera conjuntos disjuntos de variables $X,Y,Z,W$ se cumple que
		\begin{itemize}
		\item Si en el grafo $\mathcal{G}_{\bar{X}}$ se cumple que $Y$ es condicionalmente independiente de $Z$ dados $X$ y $W$, entonces
		\[ P_{\mathcal{G}}(Y=y | do(X=x), Z=z, W=w) = P_{\mathcal{G}}(Y=y | do(X=x), W=w). \]
		\item Si en el grafo $\mathcal{G}_{\bar{X}\underline{Z}}$ se cumple que $Y$ es condicionalmente independiente de $Z$ dados $X$ y $W$, entonces
		\[ P(Y=y | do(X=x), do(Z=z), W=w) = P(Y=y | do(X=x), Z = z, W=w). \]
		\item Sea $Z(W)$ el conjunto de nodos en $Z$ tales que no son ancestros de ningún nodo en $W$ en el grafo $\mathcal{G}_{\bar{X}}$. Si en el grafo $G_{\bar{X}, \bar{Z(W)}}$ se cumple que $Y$ es condicionalmente independiente de $Z$ dados $X$ y $W$ entonces,
		\[ P(Y=y | do(X=x), do(Z=z), W=w) = P(Y=y | do(X=x), W=w). \]
		\end{itemize}
		\end{teo}
		
		La interpretación de estos puntos, en palabras de \cite{pearl2009causality} se sigue intepretar la acción $do(X=x)$ como la imposición de un nuevo mecanismo el cual induce un sub-modelo caracterizado por $\mathcal{G}_{\bar{X}}$. El primer punto se obtiene al considerar que eliminar ecuaciones en un sistema no introduce nuevas dependencias y muestra que la d-separación es un criterio válido para la independencia condicional en el nuevo grafo $\mathcal{G}_{\bar{X}}$. El punto dos provee una condición para que una intervención tenga el mismo efecto que una observación pasiva y el tercer punto provee una condición para poder introducir, o eliminar una intervención externa sin afectar la probabilidad de observar $Y=y$.
		\begin{teo}{\cite{peters2017elements}}\\
		Los siguientes enunciados se cumplen:
		\begin{itemize}
		\item El cálculo do es completo; es decir, para cada intervención identificable existe una manera iterativa de aplicar las tres reglas que resulta en dicha intervención (\cite{huang2006pearl}, \cite{shpitser2006identification})
		\item Existe un algoritmo que es capaz de encontrar todas las intervenciones identificables (\cite{tian2002}, \cite{huang2006pearl})
		\item Existe un criterio necesario y suficiente para la identificabilidad de las distribuciones de intervención (\cite{shpitser2006identification}, \cite{huang2006pearl}).
		\end{itemize}
		\end{teo}
		Las reglas del cálculo do proveen una solución al problema de la identificabilidad:
		\begin{cor} Solución al problema de la identificabilidad
		Una distribución $q=P(y_1,...,y_k | do(x_1),...,do(x_n))$ es identificable en un modelo gráfico causal $\mathcal{G}$ si existe una secuencia finita de transformaciones, donde cada una de las cuales corresponde a una de las reglas del teorema \ref{docalculus} que reduce $q$ a una expresión de probabilidad condicionada sólo en datos observacionales (i.e. no aparece el operador $do$)
		\end{cor}
	\subsection{Toma de decisiones}
		\subsubsection{Teorema de von Neumann-Morgenstern: existencia de utilidad}
		El Teorema de Von Neumann-Morgenstern (\cite{von1944theory}) surge en el contexto de Teoría de Juegos, en donde otros resultados de los mismos autores exigen que un jugador busque maximizar su ganancia esperada, lo cual requería una justificación teórica (\cite{gilboa2009decision})
		Sea $X$ un conjunto finito de \textit{alternativas} y consideremos una relación de preferencias $\succeq$ sobre $X$, definimos el conjunto de \textit{loterías} sobre elementos de $X$ de la siguiente manera:
\[ \mathcal{L} = \{ P:X \to [0,1] | \sum_{x \in X} P(x) = 1, \textrm{ $P$ tiene soporte finito} \}. \]
Es decir, una lotería es una distribución de probabilidad sobre elementos de $X$. Para $P \in \mathcal{L}$ y $x \in X$. Denotamos $P(x)$ como la probabilidad que la lotería $P$ asigna a la alternativa $x$. Una forma útil de representar loterías es la siguiente (\cite{sucar2015probabilistic}, \cite{shoham2008multiagent}):

\[ [p_1: x_1, ... , p_k : o_k]; \textrm{  }, x_i \in X, p_i \geq 0, \sum_{i=1}^k  p_i = 1.\]

Podemos extender la relación de preferencias $\succeq$ que está definida sobre $X$ a $\mathcal{L}$ de modo que un tomador de decisiones puede escoger entre alternativas que son elementos de $X$ o loterías de $\mathcal{L}$. En particular, notemos que $X \subseteq \mathcal{L}$ pues un elemento $x \in X$ corresponde a la lotería $[1:x]$. Dicho esto, podemos considerar sólo relaciones de preferencia sobre $\mathcal{L}$. Para cada $P, Q \in \mathcal{L}$ y $\alpha \in [0,1]$ definimos la mezcla de loterías como
\[ (\alpha P + (1- \alpha)Q)(x) = \alpha P(x) + (1-\alpha)Q(x). \]
El Teorema de von Neumann-Morgenstern requiere de los siguientes axiomas sobre la relación de preferencias $\succeq$ que está definida sobre $\mathcal{L}$:
\begin{itemize}
\item A1. Órden débil: $\succeq$ es completa y transitiva
\item A2. Continuidad: Para toda $P,Q,R \in L$ tales que $P \succ Q \succ R$ existen $\alpha$ y $\beta$ tales que:
\[  \alpha P + (1-\alpha) R \succ Q \succ \beta P + (1-\beta) R\]
\item A3. Independencia: Para toda lotería $P,Q,R \in L$ y toda $\alpha \in (0,1)$ se tiene que $P \succeq Q$ si y sólo si 
\[\alpha P + (1-\alpha) R \succeq \alpha Q  + (1-\alpha) R.\]
\end{itemize}
\begin{teo}{según la formulación de \cite{jensen1967introduction}}\\
Una relación de preferencias $\succeq$ definida sobre un conjunto $\mathcal{L}$ de loterías definidas sobre un conjunto finito de alternativas $X$ satisface los axiomas A1 - A3 si y sólo sí existe una función $u: X \to \mathbb{R}$ tal que para toda $P,Q \in \mathcal{L}$ se tiene que $P \succeq Q$ si y sólo si 
\[ \sum_{x \in X} P(x) u(x) \geq \sum_{x \in X} Q(x) u(x).  \]
\end{teo}
Este Teorema dice que para un tomador de decisiones racional (es decir, que se comporta según los axiomas) existe una función de utilidad tal que las preferencias del tomador de decisiones son equivalentes a maximizar una utilidad esperada.
		\subsubsection{Teorema de DeFinetti: existencia de probabilidad subjetiva}
		El Teorema de DeFinetti considera el caso opuesto: consideremos un conjunto de  $n$ \textit{estados}. Una \textit{apuesta} sobre este conjunto de estados es una función que mapea estados a un número real, y podemos identificar cada una de estas apuestas como un vector de tamaño $n$, cuya $i$-ésima entrada denota el valor que se asigna al estado $i$. Denotemos por $X$ el conjunto de apuestas sobre estados del mundo y consideramos una relación $\succeq$ definida sobre $X$. De la misma manera que el Teorema de Von Neumann, requerimos de axiomas sobre la relación:
\begin{itemize}
\item A1. Órden débil: $\succeq$ es completa y transitiva.
\item A2. Continuidad: Para cada $x \in X$, los conjuntos $\{ y | x \succ y \}$ y $\{ y | y \succ x \}$ son abiertos en la topología estándar de $\mathbb{R}^n$.
\item A3. Aditividad: Para toda $x,y,z \in X$, $x \succeq y$ si y sólo si $x+z \succeq y+z$.
\item A4 Monotonicidad: Para toda $x,y \in X$, si se cumple que $x_i \geq y_i$ entonces se tiene que $x \succeq y$.
\item A5 No-trivialidad: Existen $x,y \in X$ tales que $x \succ y$.
\end{itemize}
Notemos que el axioma A3 sólo aplica si el tomador de decisiones es neutral al riesgo (\cite{gilboa2009decision}).
\begin{teo}{\cite{definetti1937}}\\
La relación $\succeq \subseteq X \times X$ satisface A1-A5 si y sólo si existe un vector de probabilidad $p$ de tamaño $n$ tal que para toda $x,y \in X$ se cumple que
\[ x \succeq y \textrm { si y sólo si } p^t x \geq p^t y. \]
\end{teo}
Este Teorema arroja una probabilidad para cada tomador de decisiones tal que la relación de preferencias del éste sea equivalente con la utilidad esperada bajo esta distrbución; por esto, a tal distribución se le conoce como \textit{probabilidad subjetiva}.
		\subsubsection{Teorema de Savage: existencia de utilidad y probabilidad subjetiva}
	    Los resultados anteriores parecen estar en polos opuestos, pues el Teorema de De Finetti provee una distribución de probabilidad para tomar decisiones con ella si el tomador de decisiones tiene una manera de medir su utilidad; por otro lado, el Teorema de Von Neumann supone que se conocen las probabilidades y otorga una forma de medir utilidad para tomar decisiones. El gran logro del Teorema de Savage es garantizar la existencia tanto de probabildad subjetiva como de una función de utilidad y además hacerlo en tándem (\cite{gilboa2009decision}).\\
	    \\
	    El Teorema de Savage requiere dos conceptos básicos: un conjunto de estados $S$ y un conjunto de resultados $X$ y las opciones a escoger en este contexto son \textit{actos}, que son funciones de los estados a los resultados:
	    \[ F = X^S = \{ f | f: S \to X \}. \]
	    Consideramos una relación binaria $\succeq$ definida sobre $F$.
	    Diremos que un conjunto $A \subseteq S$ es un evento, y no se requieren suposiciones en cuanto a la medibilidad de estos.
	    Consideremos la siguiente notación: para actos $f,g \in F$, y un evento $A \subseteq S$ definimos un acto $f_A^g$ como sigue:
	    \[ f_A^g(s)=g(s) \textrm{ si  } s \in A, \textrm{  } f(s) \textrm{ si  } s \in A^c   \]
	    Es decir, $ f_A^g$ es $g$ para valores en $A$. Con esta notación, será más fácil expresar los axiomas que este Teorema requiere:
	    \begin{itemize}
	    \item P1. Existen $f,g$ tales que $f \succeq g$
	    \item P2. Órden débil: $\succeq$ es un órden débil.
	    \item P3. Para toda $f,g,h,h' \in F$ y todo $A \subseteq S$,
	    \[ f_{A^c}^h \succeq g_{A^c}^h  \textrm{ si y sólo si }  f_{A^c}^{h'}  \succeq g_{A^c}^{h'} .\]
	    \item P4. Para toda $f \in F$, para $A \subseteq$ no-nulo y para $x,y \in X$,
	    \[ x \succeq y \textrm{ si y sólo si } f_A^x \succeq f_A^y.  \]
	    donde un evento nulo $A$ es aquel evento tal que la restricción de $\succeq$ a $A$ cumple que $f \sim_A g$ para toda $f,g \in F$.
	    \item P5. Para todo $A,B \subseteq S$ y toda $x,y,z,w \in F$ tales que $x \succeq y$, $z \succeq w$, se cumple que
	    \[ y_A^x \succeq y_B^x \textrm{ si y sólo si } w_A^z \succeq w_B^z.\]
	    \item P6. Para toda $f,g,h \in F$ tales que $f \succ g$ existe una partición $\{ A_1, ... , A_n \}$ de $S$ tal que para toda $ i \leq n$,
	    \[ f_{A_i}^h \succ g, \textrm{ y además } f \succ g_{A_i}^h. \]
	    \item P7. Para toda $f,g \in F$ y eventos $A \subseteq S$, si para toda $s \in A$ se tiene que $f \succeq_A g(s)$, entonces $f \succeq_A g$ y si para toda $s \in A$, $g(s) \succeq_A f$ entonces $g \succeq_A f$
	    \end{itemize}
	    \begin{teo}{\cite{savage1954the}}\\
	    Para un conjunto $X$ finito de resultados y un conjunto $S$ de estados. Una relación de preferencias $\succeq$ definida sobre $F=X^S$ sastisface los axiomas P1-P7 si y sólo si existe una medida $\mu$ definida sobre el espacio medible $(S, 2^S)$ que es no-atómica y finitamente aditiva, y existe una función no-constante $u :X \to \mathbb{R}$ tal que para toda $f,g \in F$,
	    \[ f \succeq g \textrm{ si y sólo si } \int_S u(f(s)) d \mu(s) \geq \int_S u(g(s)) d \mu(s). \]
	    \end{teo}
	    El caso para conjuntos de resultados generales:
	    \begin{teo}
	    Para un conjunto $X$ (no necesariamente finito) de resultados y un conjunto $S$ de estados. Una relación de preferencias $\succeq$ definida sobre $F=X^S$ sastisface los axiomas P1-P7 si y sólo si existe una medida $\mu$ definida sobre el espacio medible $(S, 2^S)$ que es no-atómica y finitamente aditiva, y existe una función no-constante $u :X \to \mathbb{R}$ tal que para toda $f,g \in F$,
	    \[ f \succeq g \textrm{ si y sólo si } \int_S u(f(s)) d \mu(s) \geq \int_S u(g(s)) d \mu(s). \]
	    \end{teo}
	\subsection{Teoría de Juegos}
	Teoría de Juegos es un área de las matemáticas cuyo objetivo es dar las herramientas necesarias para analizar la interacción estratégica de varios tomadores de decisiones (\cite{osborne1994course}). En esta sub-sección veremos varios tipos de juegos posibles. Por el momento, sólo veremos distintas clases de juegos para mostrar que el problema a tratar en esta propuesta cuenta con la maquinaria suficiente para ser modelado, pero se dejarán de lado otras cuestiones importantes de Teoría de Juegos como son el Equilibrio de Nash y las Estrategias Mixtas. 
		\subsubsection{Comportamiento Racional}
		Vamos a asumir que los agentes involucrados tienen preferencias racionales en el sentido de los axiomas de Decisión que fueron mencionados.
		\subsubsection{Juegos estratégicos y la forma normal}
		Consideremos una situación en la que dos o más agentes interactúan, y donde cada uno de ellos busca lograr un objetivo. Cada agente tomará una y sólo una decisión y de manera simultánea a los demás, a esta clase de situaciones se le conoce como Juegos Estratégicos.
		\begin{defi}{Juego estratégico en forma normal (\cite{osborne1994course}, \cite{shoham2008multiagent}) }\\
		Un juego finito en forma normal es una tupla $(N,A,u)$ donde:
		\begin{itemize}
		\item $N$ es un conjunto finito de jugadores.
		\item $A= A_1 \times \cdots \times A_n$ donde cada $A_i$ son las acciones disponibles para el jugador $i$. Cada elemento $a = (a_1,...,a_n) \in A$ es llamado un perfil de acción.
		\item Para cada jugador $i \in N$, una relación de preferencias $\succeq_i$ definida sobre $A$.
		\end{itemize}
		\end{defi}
		Notemos que por los Teoremas de Decisión que hemos visto, bajo ciertas condiciones podemos sustituir la relación de preferencias $\succeq_i$ por una función de utilidad $u_i$ para cada jugador donde $a \succeq_i b$ si y sólo si $u_i(a) \geq u_i(b)$. Notemos que podemos asociar directamente cada acción de cada jugador con un valor numérico (su utilidad y pago), lo cual permite representar el juego en forma matricial, a esta representación matricial se le conoce como la Forma Normal. \\
		\\
		Es deseable considerar el caso en el que un jugador no conoce todas las características del otro jugador; ya sea sus acciones o sus pagos, y es este desconocimiento lo que se pretende modelar con los Juegos Estratégicos Bayesianos.
		\begin{defi}{Juego Estratégico Bayesiano en Forma Normal (\cite{osborne1994course})}\\
		Un juego estratégico Bayesiano consiste en
		\begin{itemize}
		\item Un conjunto finito de jugadores $N$
		\item Un conjunto de \textit{estados} $\Omega$.
		\item Para cada jugador: un conjuto de acciones posibles $A_i$, y como en el caso anterior definimos $A = A_1 \times \cdots \times A_n$.
		\item Para cada jugador: Un conjunto finito $T_i$ de señales que puede observar el jugador $i$.
		\item Una función de señal $\tau_i : \Omega \to T_i$.
		\item Para cada jugador: una medida de probabilidad $P_i$ sobre $\Omega$ tal que $P( \tau^{-1}_i (t_i)) >0$ para $t_i \in T_i$.
		\item Una relación de preferencias $\succeq_i$ definida sobre el conjunto de medidas de probabilidad sobre $A \times \Omega$
		\end{itemize}
		\end{defi}
		La interpretación de esta clase de juegos es que el conjunto de estados $\Omega$ contiene las descripciones de las características relevantes de los jugadores, sobre lo cual cada jugador tiene \textit{creencias a priori} que están dados por la medida $P_i$. En cada jugada, se realiza un $\omega \in \Omega$, y cada jugador observa $\tau_i (\omega)$. Si un jugador recibe la señal $t_i \in T_i$, entonces deduce que el estado verdadero del mundo se encuentra en el conjunto $\tau^{-1}_i (t_i)$, por eso el requisito de asignar probabilidad mayor a cero. El jugador actualiza sus creencias sobre $\omega \in \Omega$ a $P_i(\omega) / P_i(\tau^{-1}_i (t_i))$ si $\omega \in \Omega$ y cero en otro caso. Respecto a este tipo de juegos y su aplicabilidad en el problema del uso de relaciones causales, basta decir que lo que se pretende es que al no conocer el modelo causal verdadero, las creencias del tomador de decisiones sean sobre este modelo causal, estas creencias serán utilizadas por el agente como un modelo \textit{propio} que irá actualizando conforme el mundo revela algo de sí.
		\subsubsection{Juegos extensivos con información perfecta}
		\subsubsection{Juegos extensivos con información incompleta}
		
	

\section{Pregunta de investigación}
\section{Planteamiento y Modelado}
Suponemos un ambiente que está regido por un modelo causal, y suponemos que este es desconocido, pero fijo y además cumple que todas las variables se encuentran contenidas en él (suficiencia causal).\\
\\
\indent Un problema de decisión bajo incertidumbre (de una etapa) consiste en un conjunto de acciones, un conjunto de consecuencias y un conjunto de \textit{eventos inciertos}; la interpretación estándar (\cite{bernardo2000bayesian}, \cite{gilboa2009decision}) es que el agente escogerá una de estas acciones, luego ocurrirá uno de estos eventos inciertos y será esto lo que determine la consecuencia. Suponer que un modelo causal exista y opere sobre este ambiente implica que éste controle la relación entre los eventos inciertos y las consecuencias de éste, lo cual queda codificado en probabilidades; de este modo, si se conoce el modelo causal, se conocen las probabilidades de ocurrencia de las consecuencias dada la acción escogida y el evento incierto. De esta manera, se puede determinar la acción óptima (o serie de acciones) mediante la maximización de la utilidad esperada, pues este es el único criterio de solución que es coherente con los axiomas de decisión de \cite{savage1954the}.\\
\\
\indent Cuando el modelo causal no es conocido por el tomador de decisiones, tampoco lo son las probabilidades de los efectos dadas las causas, por lo que no se puede calcular la utilidad esperada. Entonces, el camino que se propone es aprender los parámetros a partir de la información que provee el ambiente posterior a cada interacción con este. Sabemos que un problema de decisión bajo incertidumbre puede resolverse con algunas de las técnicas ya conocidas en el área de Aprendizaje por Refuerzo (\cite{sutton1998reinforcement}), resulta además que una política óptima aprendida con estas técnicas alcanza la MUE (\cite{webb2007game}). Lo que estas técnicas ignoran es cuando el agente sabe que su entorno tiene cierta estructura particular, y se desea utilizar esta información para aprender los parámetros del modelo y tomar decisiones de manera correcta. En este caso, la estructura es la que está dada por las relaciones causa-efecto en el ambiente.\\
\\
\indent Para llevar a cabo el proceso aprendizaje dotamos al agente con  \textit{creencias} sobre la estructura causal de su entorno las cuales se irán actualizando conforme el agente interactue con éste. Estas creencias pueden considerarse como un modelo causal \textit{propio} del agente, pues el agente utilizará este modelo causal para la toma de decisiones, que por construcción irá acercándose cada vez más al modelo causal verdadero. El problema de la actualización de creencias puede estudiarse en dos casos distintos: si el número de variables en el modelo causal es conocido por el agente, entonces podemos representar el modelo causal mediante matrices de tamaño fijo y llevar a cabo actualizaciones en las entradas de estas; pero si el número de variables es desconocido por el tomador de decisiones, un camino a seguir es utilizar un enfoque Bayesiano no-paramétrico y considerar distribuciones sobre un espacio de grafos; este problema parece tener una complejidad alta. Enfoques Bayesianos no-paramétricos para descubrimiento causal han sido explorados en \cite{karabatsos2012bayesian}.\\
\\
\indent Para modelar la interacción del tomador de decisiones con su entorno, vamos a asumir que el agente está interactuando con un ente abstracto al que le llamaremos Naturaleza y que la dinámica de interacción consiste en un \textit{juego}. A diferencia de lo que se hace en Aprendizaje por Refuerzo en donde se considera que el agente está “solo con su entorno”, aquí consideramos que todo lo que proviene de afuera del agente son jugadas de este ente abstracto. En el contexto de Teoría de Juegos, los juegos que permiten que el agente tenga creencias información incompleta se llaman \textit{Juegos Estocásticos}, los cuales fueron introducidos por \cite{shapley1953stochastic} y generalizan los Procesos de Decisión Markovianos (MDP), así como los juegos repetido, pues un MDP es simplemente un juego estocástico con un solo jugador mientras que un juego repetido es un juego estocástico de una sola etapa (\cite{shoham2008multiagent}).\\
\\
\indent En resumen, el modelo consiste en traducir un problema de toma de decisiones bajo incertidumbre en un juego Bayesiano entre el tomador de decisiones y la Naturaleza donde esta última va a muestrear sus acciones a partir del modelo causal que opera en el entorno y que establece las relaciones causales entre variables así como entre acciones y variables. Queremos que a lo largo de las jugadas (o de repeticiones de los juegos) el jugador vaya actualizando sus creencias causales y descubriendo ese modelo causal; es decir, se necesita que las creencias actualizadas del jugador converjan al verdadero modelo. Además, utilizando el concepto de \textit{estrategias mixtas} el jugador puede tomar ciertas acciones de manera aleatoria, lo que introduce el concepto de \textit{exploración} proveniente del área de RL. Lo que esperamos tener al final de un cierto número de rondas de aprendizaje es un \textit{diagrama de influencia causal}, que consistirá en la extensión al caso causal de los diagramas de influencia tradicionales.\\
\\
\indent Un primer planteamiento sería considerar que las acciones de la naturaleza son completamente \textit{desinteresadas}; es decir, que la naturaleza no tiene ningún objetivo en particular. En \cite{eberhardt2008causal} el problema de descubrimiento causal es planteado también como un juego, en el cual la naturaleza escoge desde el inicio el modelo causal \textit{real} y el jugador debe escoger intervenciones a llevar a cabo con tal de descubrir el modelo causal. En este caso el objetivo es descubrir el modelo causal, no el aspecto de toma de decisiones.
\section{Trabajo Previo}

\section{Objetivos}
	\section{Objetivo General (debe ir relacionado con el título)}
	\section{Objetivos específicos}
	
\section{Metodología}
	\subsection{Hipótesis}
	Tentativa: La incorporación efectiva de información causal en un proceso de toma de decisiones bajo incertidumbre produce un mejora en el desempeño del tomador de decisiones.
	Otra opción: Bajo ciertas condiciones, un agente inteligente puede adquirir y utilizar conocimiento causal a partir de su interacción con un entorno incierto.
	\subsection{Metodología de Trabajo}

\section{Resultados Preliminares}
Para mostrar la factibilidad del modelo planteado, consideramos un escenario particular y tres casos por separado cuyo grado de dificultad es ascendente. El escenario es el de un médico que tiene que escoger uno de entre varios tratamientos para un paciente con enfermedad desconocida. Los tratamientos posibles conllevan riesgos y ventajas, que están regidos por un modelo causal.\\
\\
Los tres casos a considerar son los siguientes:\\
\\
En el primer caso, supondremos que el modelo causal es conocido completamente por el agente y veremos cuál sería su manera de actuar en este caso. Que el modelo sea conocido completamente significa que el agente conoce la estructura del grafo así como las probabilidades asociadas a cada variable.\\
\\ 
Posteriormente, en el segundo caso, el agente sólo tendrá a su disposición la estructura del grafo, pero no los parámetros de este. \\
\\
En el tercer caso, el agente no conocerá nada del modelo, pero tendrá \textit{creencias} sobre este; este caso generaliza los anteriores, pues en el primer y segundo caso las creencias con las que inicia el agente es la información que conoce sobre el modelo. En resumen, en los tres casos el agente contará con \textit{creencias} sobre el modelo causal verdadero; en el primer caso, sus creencias coincidirán plenamente con el modelo; en el segundo caso, sólo con la estructura y en el tercer caso, no tenemos garantía de nada. 
\subsection{Escenario}
Consideremos un paciente que llega a un hospital y puede tener la enfermedad $A$ o la enfermedad $B$, lo cual es desconocido para el médico que lo atiende, éste tiene a su disposición tres opciones: mandar el paciente a cirugía, darle un fármaco, o no hacer nada, con lo cual el paciente morirá casi con certeza. Cada una de estas opciones entraña un riesgo: durante la cirugía el paciente puede morir, y al recibir el fármaco el paciente puede tener una reacción alérgica y morir. Aun si el paciente sobrevive, no está garantizado con certeza que el procedimiento lo cure, sino que esto depende de cuál enfermedad tenía. La cirugía es más probable que cure la enfermedad $A$ y el fármaco la $B$. Además, que el fármaco o la cirugía no tengan efecto es equivalente en términos de probabilidades a no hacer nada. 
\subsection{Caso en el que se conoce el modelo completo}
Consideremos el caso en el que el modelo gráfico causal es completamente conocido por el tomador de decisiones (el médico en este caso), tanto en su estructura como en sus parámetros. En este caso, el juego consistiría en lo siguiente:\\
\begin{itemize}
\item Un modelo gráfico causal conocido para el médico.
\item Jugadores: Naturaleza y el médico.
\item Acciones: Las acciones de la Naturaleza consisten en asignar cierto estado incierto; en este caso, asigna inicialmente la enfermedad del paciente; posteriormente, selecciona sus acciones a partir del modelo causal.
\end{itemize}
Y el juego ocurre de la siguiente forma:
\begin{itemize}
\item Primero, la Naturaleza escoge la enfermedad del paciente, pero esto el médico no lo ve.
\item Debido a que el médico no sabe qué enfermedad escogió la naturaleza, el médico se encuentra en un \textit{conjunto de información}; es decir, los nodos de decisión en los que tiene que escoger son indistinguibles. Aquí, debe escoger el médico qué acción llevar a cabo. 
\item Ahora, a partir de la elección del médico, la naturaleza muestrea del modelo causal su siguiente jugada, lo que dará la consecuencia de la acción del médico; es decir, ahora la Naturaleza escoge (a partir del modelo causal) si al paciente le da alergia y muere, si no le da alergia y se cura, si no le da alergia y muere, etc.
\end{itemize}
Debido a que el médico tiene a su disposición el modelo causal desde un inicio, puede consultar las probabilidades de $P( \textrm{vivir} | do(\textrm{procedimiento}))$ y escoger como estrategia el procedimiento que arroje la probabilidad más alta; de esta manera, el médico estaría escogiendo su \textit{mejor respuesta} a cualquiera de las jugadas de la naturaleza por lo que esta sería una solución de equilibrio. Notemos que para tener una solución de equilibrio también las jugadas de la naturaleza deben ser la mejor respuesta a cualquiera de las acciones posibles del médico,  pero vamos a darle a la naturaleza el mismo pago en todos los resultados posibles para no entrar en discusión de las \textit{intenciones} de la naturaleza. Además, darle a la Naturaleza utilidad en los pagos permite dejar espacio para posibles generalizaciones futuras; por ejemplo, en el caso en el que la Naturaleza sí tenga algún tipo de intenciones o de objetivos a cumplir.
\\
% Node styles
\tikzset{
% Two node styles for game trees: solid and hollow
solid node/.style={circle,draw,inner sep=1.5,fill=black},
hollow node/.style={circle,draw,inner sep=1.5}
}
\begin{tikzpicture}[scale=1.5,font=\footnotesize]
% Specify spacing for each level of the tree
\tikzstyle{level 1}=[level distance=15mm,sibling distance=40mm]
\tikzstyle{level 2}=[level distance=15mm,sibling distance=15mm]
\tikzstyle{level 3}=[level distance=15mm,sibling distance=11mm]
% The Tree
\node(0)[solid node,label=above:{$N$}]{}
	child{node(1)[solid node]{}
		child{node[hollow node,label=left:{$N$}]{} 
			child{node[hollow node,label=below:{$m$}]{}} 
			child{node[hollow node,label=below:{$v$}]{}}
		edge from parent node[left]{$f$}}	
		child{node[hollow node,label=below:{$N$}]{} 
			child{node[hollow node,label=below:{$m$}]{}} 
			child{node[hollow node,label=below:{$v$}]{}}
		edge from parent node[right]{$c$}}
		child{node[hollow node,label=below:{$N$}]{} 
			child{node[hollow node,label=below:{$m$}]{}} 
			child{node[hollow node,label=below:{$v$}]{}}		
		edge from parent node[right]{$n$}}
	edge from parent node[left,xshift=-3]{$A$}
}
child{node(2)[solid node]{}
child{node[hollow node,label=below:{$(e,f)$}]{} edge from parent node[left]{$f$}}
child{node[hollow node,label=below:{$(g,h)$}]{} edge from parent node[right]{$c$}}
child{node[hollow node,label=below:{$(c,d)$}]{} edge from parent node[right]{$n$}}
edge from parent node[right,xshift=3]{$B$}
};
% information set
\draw[dashed,rounded corners=10]($(1) + (-.2,.25)$)rectangle($(2) +(.2,-.25)$);
% specify mover at 2nd information set
\node at ($(1)!.5!(2)$) {$M$};
\end{tikzpicture}
\subsection{Caso en el que se conoce la estructura del modelo}
Aquí se conocen los efectos de las causas, pero no sus probabilidades, entonces primero debo conocer el subconjunto posible de consecuencias y ahí tal vez hacer exploración greedy.
\subsection{Caso general}
En el caso general en el cual el tomador de decisiones no conoce el modelo causal, éste contará con una distribución de probabilidad sobre el modelo causal que represente las de \textit{creencias}, o conocimiento previo, que tiene el agente sobre el modelo; es decir, el tomador de decisiones tendrá un modelo causal \textit{propio} que usará y modificará con la experiencia. Estas creencias tendrán la forma de una distribución de probabilidad sobre algún espacio; si suponemos que el tomador de decisiones conoce al menos el número de variables del modelo, entonces las creencias serán distribuciones de probabilidad sobre matrices de tamaño fijo; en cambio, de no conocer el tamaño del modelo causal, debemos recurrir a distribuciones de probabilidad que permitan representar objetos mucho más complejos. Una opción, es recurrir a distribuciones sobre espacios de dimensión infinita, como pudiera ser el espacio de los Grafos Acíclicos Dirigidos. Estas distribuciones sobre espacios de dimensión infinita se conocen en la literatura como \textit{priors Bayesianos no-paramétricos} y existe toda un área de la Estadística que los estudia, a saber, la Estadística Bayesiana No Paramétrica (\cite{ghosal2017fundamentals}).\\
\\
\indent Además de la maquinaria matemática para representar dichas creencias, debemos considerar que el proceso de adquisición y uso de creencias debe ser coherente con los axiomas de racionalidad que estamos suponiendo, pues estas creencias afectarán el proceso de toma de decisiones, y como es deseable poder utilizar los resultados ya establecidos, es necesario que la construcción de estas creencias se mantengan en la misma línea. Además, para tener cierto grado de realismo e interpretación, es importante apegarse también a lo que dice la psicología (\cite{larrouy2017mindreading}).\\
\\
\indent Tomando esto en cuenta, una de las contribuciones esperadas de esta investigación es construir un procedimiento de actualización de creencias acerca de la estructura causal de cierto entorno sobre el cual un agente está actuando.

\section{Plan de trabajo}

\section{Conclusiones}

\section{Limitaciones}
En este trabajo consideramos un tomador de decisiones que actúa racionalmente.

Consideramos la definición de causalidad que aparece arriba y la consideramos de forma axiomática; es decir, no se pone en tela de juicio si los supuestos son realistas o no, ni es intención encontrar la definición de causalidad que mejor se adapte a una situación particular del mundo \textit{real}.

No estamos considerando que el ente que llamamos Naturaleza tenga intenciones, pero al darle \textit{pagos} en cada juego, aunque iguales, podríamos permitir que busque maximizar algún objetivo.

\section{Extensiones posibles}
Más pasos dentro de un mismo juego, oponentes, intenciones de la naturaleza

\bibliographystyle{apalike}
\bibliography{/Users/MauricioGS1/INAOE/Segundo_Semestre/Propuesta/Bibliografia.bib}
\end{document}
