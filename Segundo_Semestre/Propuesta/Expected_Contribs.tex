\documentclass[11pt]{article}
\usepackage[utf8]{inputenc}
\usepackage[spanish, mexico]{babel}
\usepackage{listings}
\usepackage{breakcites}
\usepackage{dsfont}
\usepackage{hyperref}
\usepackage{amssymb,amsthm,amsmath,latexsym}
\usepackage[margin=1.5cm]{geometry}
\usepackage{natbib}
%\bibliographystyle{stylename}
%\usepackage{fancyhdr}
%\pagestyle{fancy}
\theoremstyle{plain}
\newtheorem{teo}{Teorema}
\newtheorem{prop}[teo]{Proposición}
\newtheorem{defi}[teo]{Definición}
\newtheorem{obs}[teo]{Observación}
\newtheorem{lem}[teo]{Lema}
\newtheorem{cor}[teo]{Corolario}
\usepackage[pdftex]{color,graphicx}
\newcommand{\sgn}{\mathop{\mathrm{sgn}}}
\title{Problemas de Decisión bajo Incertidumbre: el caso de estructura causal.}
\author{Mauricio Gonzalez Soto}
\begin{document}
%\nocite{*}
\maketitle
\section{Introducción}
En múltiples ocasiones y contextos diversos los seres humanos nos vemos forzados a tomar decisiones cuyas consecuencias son inciertas. A lo más, podemos contar con estimaciones sobre lo probable que es la ocurrencia de algún evento incierto.\\
\\
La Teoría de la Decisión bajo Incertidumbre, desarrollada en los trabajos de \cite{von1944theory}, \cite{definetti1930}, \cite{definetti1937}, \cite{savage1954the}, \cite{bernardo2000bayesian}, define un criterio formal de elección, el cual consiste en la maximización de la utilidad esperada.\\
\\
En diversas ocasiones, un tomador de decisiones no tiene a su alcance todos los parámetros, pero puede \textit{aprenderlos} del mismo entorno, sobre lo cual los trabajos de Aprendizaje por Refuerzo (\cite{sutton1998reinforcement}) muestarn cómo. Un caso particular de eventos inciertos es cuando el tomador de decisiones sabe con certeza que en el ambiente en el cual se llevan a cabo sus decisiones existen ciertos aspectos \textit{estructurales} que pueden ser incorporados en la toma de decisiones; el caso en el cual nos vamos a centrar es en el caso en el cual existen relaciones causa-efecto en el ambiente.\\
\\
Las relaciones causa-efecto son de gran utilidad en problemas de decisión, pues conociendo los \textit{efectos} de las posibles decisiones a tomar, el agente puede planear mejor sus acciones (\cite{hagmayer2013repeated}).\\
\section{Problema de investigación}
Adqusición y uso de información causal en problemas de toma de decisiónes secuenciales bajo incertidumbre.
\section{Planteamiento del Problema}
Supondremos que se conoce de antemano cuando un ambiente está regido por un modelo causal, y suponemos que este es desconocido, pero fijo.\\
\\
Un problema de decisión bajo incertidumbre (de una etapa) consiste en un conjunto de acciones, un conjunto de consecuencias y un conjunto de \textit{eventos inciertos}; la interpretación estándar es que el agente escogerá una de estas acciones, luego ocurrirá uno de estos eventos inciertos y será esto lo que determine la consecuencia. Suponer que un modelo causal exista y opere sobre este ambiente implica que éste controle la relación entre los eventos inciertos y las consecuencias de éste, lo cual queda codificado en probabilidades; de este modo, si se conoce el modelo causal, se conocen las probabilidades de ocurrencia de las consecuencias dada la acción escogida y el evento incierto. De esta manera, se puede determinar la acción óptima (o serie de acciones) mediante la maximización de la utilidad esperada, pues este es el único criterio de solución que es coherente con los axiomas de decisión de \cite{savage1954the}.\\
\\
En cambio, cuando no se conoce el modelo causal, no se puede utilizar la utilidad esperada para encontrar el camino a seguir. Entonces, se podría intentar aprender los parámetros a partir de la información que provee el ambiente posterior a cada interacción con este. Sabemos que un problema de decisión bajo incertidumbre puede resolverse con algunas de las técnicas ya conocidas en el área de Aprendizaje por Refuerzo (\cite{sutton1998reinforcement}), resulta además que una política óptima aprendida con estas técnicas alcanza la MUE (\cite{webb2007game}). Lo que estas técnicas no toman en cuenta es cuando el agente sabe que su entorno tiene cierta estructura particular, y se desea utilizar esto para aprender los parámetros.\\
\\
Para esto, queremos dotar al agente con \textit{creencias} sobre la estructura de su entorno; entonces, vamos a pensar que el agente está interactuando con otro ente abstracto (llamémosle Naturaleza), y esta interacción la vamos a modelar como un \textit{juego}.\\
\\
Teoría de Juegos (\cite{osborne1994course}) es un área de las matemáticas que los economistas y otros científicos sociales utilizan para modelar situaciones de interacción estratégica; es decir, problemas de decisión en los cuales importa no solo las acciones y preferencias de un sólo agente sino las de otros. En particular, en esta área existen los llamados \textit{juegos Bayesianos} en los cuales cada jugador tiene creencias \textit{a priori} sobre algunos aspectos del ambiente (en las aplicaciones de economía estas creencias suelen versar sobre las recompensas o sobre las estrategias de otros jugadores). En general, los juegos Bayesianos pertenecen a una clase de juegos llamados juegos estratégicos.\\
\\
La clase de juegos necesaria para modelar un proceso de decisiones secuenciales bajo incertidumbre y con información incompleta se llaman \textit{Juegos Estocásticos}, que de hecho generalizan los Proceso de Decisión Markovianos así como los juegos repetiso, pues un MDP es simplemente un juego estocástico con un solo jugador mientras que un juego repetido es un juego estocástico de una sola etapa (\cite{shoham2008multiagent}).\\
\\
En nuestro caso, vamos a pensar que se está llevando a cabo un juego (Bayesiano) entre el tomador de decisiones y la Naturaleza, quien va a muestrear sus acciones a partir del modelo causal que opera en el entorno. Queremos que a lo largo de las jugadas (o de repeticiones de los juegos) el jugador vaya actualizando sus creencias causales y descubriendo ese modelo causal; es decir, se necesita que las creencias actualizadas del jugador converjan al verdadero modelo. Además, utilizando el concepto de \textit{estrategias mixtas} el jugador puede tomar ciertas acciones de manera aleatoria, lo que introduce el concepto de \textit{exploración} proveniente del área de RL. Lo que esperamos tener al final de un cierto número de rondas de aprendizaje es un \textit{diagrama de influencia causal}, que consistirá en la extensión al caso causal de los diagramas de influencia tradicionales.\\
\\
Un primer planteamiento sería considerar que las acciones de la naturaleza son completamente \textit{desinteresadas}; es decir, que la naturaleza no tiene ningún objetivo en particular.En \cite{eberhardt2008causal} el problema de descubrimiento causal es planteado también como un juego, en el cual la naturaleza escoge desde el inicio el modelo causal \textit{real} y el jugador debe escoger intervenciones a llevar a cabo con tal de descubrir el modelo causal. En este caso el objetivo es descubrir el modelo causal, no el aspecto de toma de decisiones.
\section{Trabajo Previo}
Existen trabajos que utilizan en conjunto inferencia causal junto con toma de decisiones; como por ejemplo \cite{heckerman1995decision} quien argumenta que los trabajos usuales sobre causalidad no contienen primitivas suficientemente básicas, por lo que plantea el problema de inferencia causal como uno de toma de decisiones.\\
\\
Por otro lado, \cite{dawid2002influence} explica cómo plantear un problema de inferencia causal utilizando diagramas de influencia, los cuales son una extensión de las redes Bayesianas para incluir nodos de decisión, y los cuales pueden utilizarse para descubrir las acciones óptimas en un problema de decisión tradicional (sí, aquella serie de acciones que obtiene la máxima utilidad esperada). El interés del autor es utilizar la representación de diagramas de influencia para \textit{interrogar} el fenómeno. En nuestro caso, el objetivo es \textit{construir} este diagrama mediante un ciclo de decisión-observación.\\
\\
\cite{koller2003multi} utilizan diagramas de influencia para encontrar estrategias relevantes en juegos multi-agente, con lo cual descomponen un juego grande en varios juegos más chicos que se resuelven de manera secuencial.\\
\\
\cite{lattimoreNIPS2016} atacan un problema de toma de decisiones en entornos causales; en su modelo, las acciones a escoger son intervenciones posibles en el modelo y lo que se busca es maximizar cierta variable de recompensa que está influenciada directamente por el modelo causal. La optimalidad de su acción está determinada en términos del \textit{regret} mínimo, que es básicamente un valor esperado. Esta medida de optimalidad no está relacionada con ninguna función de utilidad asociada a un agente y esto restringe que el mismo problema se resuelva con distintos fines en mente.\\
\\
\cite{larrouy2017mindreading} estudian la formación de creencias a partir de las acciones observadas y definen el concepto de \textit{mindreading}; es decir, poder adelantarse a las creencias que el otro jugador tomará en cuenta a la hora de tomar sus decisiones.\\
\\
La relación entre Teoría de Juegos, toma de decisiones e incorporación de información adicional (causal) parece muy clara, aunque no ha sido particularmente explotada. A lo más, los trabajos plantean el problema per se de inferencia causal \textit{como} un juego. En nuestro caso, lo que se propone es jugar un juego para el cual existan relaciones causales de fondo y utilizar lo que se pueda aprender de estas.
\section{Contribuciones Esperadas}
Para llevar a cabo lo anterior se necesita:
\begin{itemize}
\item Una forma de especificar creencias sobre estructura causal; esto puede ser mediante una distribución probabilista sobre un espacio de grafos. Podemos suponer que el número de variables es conocido de antemano, con lo cual podríamos limitarnos a trabajar sobre distribuciones de matrices de tamaño fijo. Si se permite que el número de variables sea arbitrario/desconocido, tendríamos que operar con modelos Bayesianos no paramétricos.
\item Un criterio de actualización de creencias causales que además debe cumplir la propiedad de que converge al verdadero modelo causal así como ser coherente con estudios de la psicología de la formación de creencias (\cite{larrouy2017mindreading}).
\item Un criterio de solución del problema: Si no es posible esperar hasta la convergencia del modelo, sí debemos garantizar que las decisiones tomadas con el modelo causal aprendido hasta cierta ronda aproximen la utilidad máxima esperada.
\item Una manera de construir un diagrama de influencia causal que permita \textit{exportar} el conocimiento adquirido para problemas similares; de esta manera, ya sabríamos directamente cómo actuar conociendo el modelo causal.
\end{itemize}
\section{Ejemplo}
Supongamos que un paciente que llega a un hospital puede tener un padecimiento $A$ con probabilidad $p$ o $B$ con probabilidad $q$. El médico de guardia tiene a su alcance tres tratamientos posibles: cirugía, fármaco o nada. En caso de no hacer nada, el paciente morirá con una probabilidad $p_{dn}=0.8$ y sobrevivirá con una probabilidad $1-p_d$. Sin importar la enfermedad, el paciente puede morir durante la cirugía con probabilidad $p_{dc}=0.5$. De la misma manera, el medicamento producirá una reacción alérgica mortal con probabilidad $p_{df}=0.2$\\
\\
Si el paciente tenía la enfermedad $A$, y no muere de una reacción alérgica al fármaco, entonces el fármaco le curará con una probabilidad de $p_{sf}=0.9$. Si el paciente tenía la enfermedad $B$, entonces el fármaco no le curará y morirá.\\
\\
Por otro lado, si el paciente sobrevive a la cirugía, ésta le curará con una probabilidad de $0.5$ si el paciente tenía la enfermedad $A$. Si el paciente tenía la enfermedad $B$, entonces ésta lo curará con probabilidad de $0.6$. \qed
\\
\\
Entonces, en este ejemplo, tenemos que el juego consiste en el médico, y en la Naturaleza. Las acciones disponibles al médico son el fármaco, cirugía, o nada, mientras que las jugadas de la Naturaleza son \textit{volados} que determinarán si el paciente vive o no; el juego va así, primero la naturaleza hace su jugada, lo cual determina qué enfermedad tenía el paciente. Después, el médico decide qué hacer, y después la Naturaleza lanzará una moneda cargada que determinará si el paciente sobrevive. Notemos que, de manera latente, es un modelo causal el que determina que el fármaco no cura la enfermedad B, o que éste causa alergias, etc.\\
\\
Sabemos que el médico fue a la escuela de medicina, de donde posee una gran cantidad de información (causal) a priori, pero puede que se esté enfrentando a una nueva enfermedad, es decir, no conoce realmente las probabilidades de muerte ni los casos en que el fármaco no actua. Lo que se desea es que a partir de este caso, o de varios más, pueda actualizar su conocimiento causal para la siguiente ocasión en la que deba enfrentarse a  un problema similar.
\bibliographystyle{apalike}
\bibliography{Bibliografia}
\end{document}