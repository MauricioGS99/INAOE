\documentclass[11pt]{article}
\usepackage[utf8]{inputenc}
%\usepackage[spanish, mexico]{babel}
\usepackage[font=small,labelfont=bf]{caption}
\usepackage{listings}
\usepackage{breakcites}
\usepackage{dsfont}
\usepackage{hyperref}
\usepackage{amssymb,amsthm,amsmath,latexsym}
\usepackage[margin=1.5cm]{geometry}
\usepackage{natbib}
%\bibliographystyle{stylename}
%\usepackage{fancyhdr}
%\pagestyle{fancy}
\theoremstyle{plain}
\newtheorem{teo}{Theorem}
\newtheorem{prop}[teo]{Proposition}
\newtheorem{defi}[teo]{Definition}
\newtheorem{obs}[teo]{Observation}
\newtheorem{lem}[teo]{Lemma}
\newtheorem{cor}[teo]{Corolary}
\usepackage[pdftex]{color,graphicx}
\usepackage{tikz}
\usetikzlibrary{trees}
\usetikzlibrary{calc}
%\newcommand{\sgn}{\mathop{\mathrm{sgn}}}
\title{Use and acquisition of causal relations for decision making under uncertainty using imperfect information repeated games}
\author{Mauricio Gonzalez Soto}
\begin{document}
%\nocite{*}
\maketitle
\tableofcontents
\newpage
\begin{abstract}
We consider decision problems under uncertainty where the options available to a decision maker and the resulting outcome are related through a causal mechanism which is unknown to the decision maker, although he is aware of the causal nature of his environment. We study how a decision maker can learn about this causal mechanism through sequential decision making as well as using current causal knowledge inside each round in order to make better choices had he not considered causal knowledge. We propose a decision making procedure in which an agent holds \textit{beliefs} about her environment which are used to make a choice and then are updated using the observed outcome. As proof of concept, we present an implementation of this causal decision making model and apply it to a simple problem. We show that the model achieves a performance similar to the classic Q-learning while it also acquires a causal model of the environment. 
\end{abstract}
\section{Introduction}
\indent A fundamental part of intelligent reasoning is being able to make decisions under uncertain conditions (\cite{danks2014unifying}, \cite{lake2017building}, \cite{pearlwhy}). In some cases, a decision maker who faces an uncertain environment has enough information to make choices by maximizing expected utility, which is the classic formal criteria for making decisions if rational preferences are assumed (\cite{bernardo2000bayesian}, \cite{gilboa2009decision}). On the other hand, if enough information is not available, the decision maker could attempt to \textit{learn} from the environment by interacting with it.

Learning by interaction has been extensively studied by computer scientists using the Reinforcement Learning (RL) setting \cite{sutton1998reinforcement}, but the most common used techniques  in this field are purely associative and do not consider any high-level structure of the environment beyond what is expressable in a Markov Decision Process \cite{garnelo2016towards}.

Since human beings are known to learn causal models in sequential decision making processes (\cite{sloman2006causal}, \cite{nichols2007decision}, \cite{meder2010observing}, \cite{hagmayer2013repeated}, \cite{danks2014unifying}), and even though this learning is not perfect \cite{rottman2014reasoning}, we propose that an autonomous agent can learn and use causal information while interacting with an uncertain environment which is governed by a fixed \textit{causal mechanism} which is unknown to the agent.  

The proposed way for an agent to learn from repeated interactions is by giving her \textit{beliefs} about the structure of the environment and a way to update them after an outcome has been observed.

While the standard setting in RL is to model the agent-environment interaction as an agent that moves from one \textit{state} to another inside a model of the environment and observing a reward as these transitions occur, we propose to model it as a \textit{game} between the decision maker and a player called \textit{Nature} which will select his actions from the causal model in response to what the decision maker has chosen. 

The agent, using her current beliefs, will generate a \textit{local} causal model and choose an action from it as if that model was the true one. Then, after she observes the consequences of her actions, her beliefs will be updated according to the observed information in order to make a better choice the next time. The agent, besides learning a policy to choose actions will also learn a causal model from the environment since the causal model she forms will approximate the true model.

Learning a causal model of the environment allows to extract high-level insights of a phenomena beyond associative descriptions of what is observed. A causal model is able to \textit{explain} why a particular decision was made since it allows to extract the causes and effects of an agent's actions.

\newpage
\section{Main Concepts}
	\subsection{Attempts to define Causality}
	Causality has been a difficult concept to define and several attempts have been made. In fact, in mainstream statistics  the term \textit{cause} have been nearly banned as documented by \cite{pearl2018why}.
	
	In the reference works on Causality (\cite{spirtes2000causation}, \cite{pearl2009causality}) the term \textit{causality} is defined in terms of common language and then the mathematical machinery required to model causality is provided. In this way, causality becomes defined as what is modeled by causal models, which are themselves defined as to model causality. This apparent circularity is deeply studied by \cite{woodward2005making}. For this reason, we will give here a formal working definition of causal relations which allows to be used formally.
	
	Several definitions of Causality exist in diverse contexts with diverse applications in mind, such as the Topological Causality for Dynamical Systems (\cite{harnack2017topological}), Lamport's Causality (\cite{lamport1978time}), Granger's Causality for Time Series (\cite{granger1969investigating}), Suppes' Causality (\cite{suppes1970probabilistic}) each with its own mathematical framework. A review of several of this theories, including Suppes' and Granger's can be found in \cite{holland1985statistics}.
	\subsection{Definition of Causality}
	Causality is understood as a \textit{stochastic} relation between \textit{events}. Some event \textit{causes} another event to occurr. For example, The formal definition of Causality that will be adopted in this work is the definition provided in \cite{spirtes2000causation}.
	\begin{defi}{\label{causal_relation}}
	Let $(\Omega, \mathcal{F}, \mathbb{P})$ a finite probability space, and consider a binary relation $\to \subseteq \mathcal{F} \times \mathcal{F}$ which is:
	\begin{itemize}
	\item Transitivity: If $A \to B$ and $B \to C$ for any $A, B, C \in \mathcal{F}$ then $A \to C$.
	\item Irreflexive: For all $A \in \mathcal{F}$ it doesn't hold that $A \to A$.
	\item Antisymmetric: For $A,B \in \mathcal{F}$ such that $A \neq B$ if $A \to B$ then it doesn't hold that $B \to A$.
	\end{itemize}
	We will say that $A$ is \textit{a cause} of $B$ (or that $A$ causes $B$, $A$ is the cause and $B$ is the effect) if $A \to B$.
	\end{defi}
	We will assume that for any event $A \in \mathcal{F}$ there are no causes lying outside $\mathcal{F}$, and that the relations expressed by $\to$ are the only causal relations in the environment. This assumption can be interpreted as not allowing \textit{intermediate} events between events known to be causally related. For example, if in our space we have that striking a match causes fire, we will not allow into consideration the underlying chemical reactions that cause fire from friction. In this sense, we will say that striking a match is a \textit{direct cause} of fire and this will be the only causes considered (\cite{spirtes2000causation}). 
	\subsection{Representation into a Directed Acyclic Graph}
	Making some notation abuse, the causal relations contained in $\to$ can be summarized in a graph $G=(V,E)$ in the following way: If $A \to B$ then the graph must contain a node $A \in V$ representing $A$, a node  $B \in V$ representing $B$ and a directed edge $e \in E$ connecting the respective nodes in the direction of the causal relation.
	\begin{prop}
	Given a causal relation $\to$ as in Definition \ref{causal_relation} then the graph that is obtained by considering nodes for events and edges for the causal relations as previously described is a Directed Acyclic Graph.
	\end{prop}
	\begin{proof}
	The graph is directed since the definition of causality imposes a direction between events; namely, a direction between the cause and the effect. To see that the graph is acyclic, suppose that a cycle $A \to B \to C \to A$ exists, this would imply, because of transitivity, that $A \to A$, which can not be since the relation is irreflexive.
	\end{proof}
	Notice that since the graph is finite, there exist some nodes that does not have causes, which are called \textit{exogenous}. If an $A$ event is caused by some other event, then we say it is \textit{endogenous} and we denote the set of its causes as $Pa(A)$.
	\subsection{Causal Graphical Models}
	A causal graphical model (CGM) consists of a set of random variables $\mathcal{X}=\{ X_1,...,X_n \}$, an acyclic directed graph (DAG) whose nodes are in correspondence with the variables in $\mathcal{X}$ and whose edges represent relations of cause-effect. Also, the model, which up to this point is nothing more than a Bayesian Network with causal semantics, is enriched with an operator named $do()$ which is defined over graphs and whose action is described as follows:
	\subsection{Decision Theory}
	A Decision Problem under Uncertainty is a situation in which an agent must choose one out of many actions with uncertain consequences. The most common theories for Decision Making under Uncertainty are those from \cite{von1944theory} and \cite{savage1954the}. In the former theory it is assumed that the decision maker knows the stochastic relation between actions and consequences, and in that case the theory guarantees that the decision maker behaves as if he maximizes the expected value of an utility function. On the other hand, if the decision maker doesn't know the probabilities of observing outcomes given a chosen action, then Savage's theory guarantees that the decision maker behaves as if he had in mind a \textit{subjective} probability distribution, a utility function and chooses the action which maximizes the expected utility with respect to that subjective probability distribution and utility function. 
	
Both von Neumann and Savage's Decision Under Uncertainty theories are known as the \textit{classic} theories in other areas, mostly in Economics. It is important to say that other Decision Making theories exist, such as Prospect Theory (\cite{kahneman1979prospect}), Case-Based Decision Theory (\cite{gilboa1995case}) among others that are out of the scope of this work.

For what remains of this section, we follow \cite{bernardo2000bayesian} in his exposition of Savage's theory since it is the most suitable for our purposes because we will be considering decision makers that do not know all of their environment. Savage's Decision Making theory, together with von Neumann's theory is what it is commonly known as \textit{classical} decision making. 
	
	\subsubsection{Elements of a Decision Problem}
	A Decision Problem under Uncertainty is composed by:
	\begin{itemize}
	\item A set $\Omega$ of \textit{states}.
	\item A set $\mathcal{A}=\{a_i : i \in I \}$ of actions.
	\item For each action $a_i$, a partition $E_i = \{ E_j : j \in J \}$ of $\Omega$. Define $\mathcal{E} = \cup_{i \in I } E_i$.
	\item For each action $a_i$, a set of consequences $C_i = \{ c_j : j \in J \}$. Define $\mathcal{C}=\cup_{i \in I} C_i$
	\item A preference relation $\succeq$ defined over $\mathcal{A}$ which represent the decision-maker whishes. It is assumed that the decision maker will choose among his options according to his preferences, which are represented by $\succeq$.
	\end{itemize}
	Since we are considering that a decision maker chooses an action $a \in \mathcal{A}$, and then an uncertain event $E$ will occur which will then produce the consequence $c$, we will identify actions in $\mathcal{A}$ as $a_i = \{ c_j | E_j : j \in J \}$. As a technical assumption required for all of the math to work, we require that elements of $\mathcal{C}$ belong to $\mathcal{A}$. This is achieved by adding into $A$ elements of the form $\{ c | \Omega \}$ for $c \in \mathcal{C}$. This being said, we can extend the preference relation $\succeq$ to elements of $\mathcal{C}$ and therefore indistinctively write $c_1 \succeq c_2$ for $c_1, c_2 \in \mathcal{C}$ or $a_1 \succeq a_2$ for $a_1,a_2 \in \mathcal{A}$. We require that $\mathcal{E}$ is an \textit{algebra}.	
	\begin{defi}
	From the prerefence relation $\succeq$ we can derive some other relations between actions:
	\begin{itemize}
	\item $a_1 \sim a_2 \Leftrightarrow a_1 \succeq a_2 \textrm{ and } a_2 \succeq a_1$
	\item $a_1 \succ a_2 \Leftrightarrow a_1 \succeq a_2 \textrm{ and it does not hold that } a_2 \succeq a_1$.
	\end{itemize}
	\end{defi}
	\begin{defi}
	A relation between events can be derived from $\succeq$ in the following way:
	\[E \succeq F \Leftrightarrow \textrm{ for all } c_1 \succeq c_2 \textrm{ it holds that } \{ c_2 | E, c_1 | E^c \} \succeq \{ c_2 | F, c_1 | F^c \}. \]
	In this case, we say that $F$ is \textbf{not more likely} than $E$. It can easily be shown that $\Omega \succ \emptyset$, and a relation $\sim$ for events is defined in an analog way than for consequences.
	\end{defi}
	\begin{defi}
	Given some $G \succ \emptyset$ we define a conditional preference relation $\succeq_G$ as follows:
	\begin{itemize}
	\item i) $a_1 \succeq_G a_2 \Leftrightarrow \textrm{ for all a } \{ a_1 | G, a | G^c \} \succeq \{ a_2 | G, a | G^c \}$.
	\item ii) $E \succeq_G F \Leftrightarrow \textrm{ for } c_1 \succeq_G c_2 \{ c_2 | E , c_1 | E^c \} \succeq_G \{c_2 | F , c_1  | F^c \}$ 
	\end{itemize}
	\end{defi}
	\begin{defi}
	Two events $E,F \in \mathcal{E}$ are said to be independent if and only if:
	\begin{itemize}
	\item 	
	\end{itemize}
	\end{defi}
	\subsubsection{Axioms of Coherence}
	We require the following axioms for the preference relation $\succeq$ which we will call Axioms of Coherence or Rationality, and define a Rational decision maker as a decision maker who will choose according to a preference relation and whose preferences satisfy the axioms (\cite{bernardo2000bayesian}, \cite{gilboa2009decision}).\\
	\\
	\indent \textbf{Axiom 1}.
	\begin{itemize}
	\item i) There exists consequences $c_1$, $c_2$ such that $c_1 \succ c_2$.
	\item ii) For all consequences $c_1, c_2$ and events $E,F \in \mathcal{E}$ either $\{  c_2 | E , c_1 | E^c\} \succeq \{ c_2 | F, c_1 | F^c \}$ or $\{ c_2 | F, c_1 | F^c \} \succeq \{  c_2 | E , c_1 | E^c\}$
	\end{itemize}
	
	\textbf{Axiom 2}.
	\begin{itemize}
	\item i) $a \succeq a$ for all $a \in \mathcal{A}$.
	\item ii) If $a \succeq b$ and $b \succeq c$ for $a,b,c \in \mathcal{A}$, then $a \succeq c$.
	\end{itemize}
	
	\textbf{Axiom 3}
	\begin{itemize}
	\item i) If $a_1 \succeq a_2$ then for all $G \succ \emptyset$ $a_1 \succeq_G a_2$.
	\end{itemize}
	
	\textbf{Axiom 4.}\\
	There exists a sub-algebra $\mathcal{S}$ of $\mathcal{E}$ and a function $\mu : \mathcal{S} \to [0,1]$ such that:
	\begin{itemize}
	\item i) $S_1 \succeq S_2$ if and only if $\mu(S_1) \succeq \mu(S_2)$.
	\item ii) If $S_1 \cap S_2 = \empty$ then $\mu(S_1 \cup S_2) = \mu(S_1) + \mu(S_2)$.
	\item iii) For any number $\alpha \in [0,1]$ and independent events $E,F$, there is an event $S$, which is called a standard event, such that $\mu(S)=\alpha$ and $S$ is independent from $E$ and from $F$.
	\item iv) If $S_1$ is independent from $S_2$ then $\mu(S_1 \cap S_2)=\mu(S_1)\mu(S_2)$.
	\item v) If $E$ independent of $S$, $S$ independent from $F$ and $E$ independent from $F$ then $E \sim S$ implies that $E \sim_F S$.
	\end{itemize}
	
	\textbf{Axiom 5.}
	\begin{itemize}
	\item i) If $c_1 \succeq c \succeq c_2$ then there exists a standard event $S$ such that $c \sim \{ c_2 | S, c_1 | S^c \}$.
	\item ii) For each event $E$ there exists a standard event $S$ such that $E \sim S$.
	\end{itemize}
	
	For interpretations and critiques of the axioms we refer the reader to \cite{bernardo2000bayesian}, \cite{binmore2008rational}, \cite{gilboa2009decision}, \cite{wakker2010prospect}, \cite{peterson2017introduction}.
	\subsubsection{Subjective Probability and Utility}
	\begin{defi}
	Given a preference relation $\succeq$, we define the (subjective) \textbf{probability} of an event $E \in \mathcal{E}$ as the real number $\mu(S)$ associated with the standard event $S$ such that $E \sim S$.
	\end{defi}
	The subjective probability thus defined satisfies all of the Kolmogorov's axioms of probability (\cite{bernardo2000bayesian}). So, we can use all of the mathematical machinery for classic probability measures, including the definition of conditional probability, Bayes' theorem, etc.
	\begin{defi}
	Consequences $c^\ast$ and $c_\ast$ are called respectively best and worst (extreme consequences) if for any other consequence $c \in \mathcal{C}$ then $c^\ast \succeq c \succeq c_\ast$. Decision problems in which we add extreme consequences are called bounded decision problems.
	\end{defi}
	\begin{defi}
	Given a preference relation $\succeq$ we define the canonical \textbf{utility} $u(c)=u(c | c_\ast, c^\ast)$ of a consequence $c \in \mathcal{C}$ relative to the extreme consequences $c_\ast$, $c^\ast$, as the real number $\mu(S)$ associated with any standard event $S$ such that $c \sim \{c^\ast | S, c_\ast | S^c\}$.
	\end{defi}
	\subsubsection{Decision Criteria for Decision Problems}
	\begin{prop}
	For any bounded decision problem with extreme consequences $c_\ast$ and $c^\ast$ it holds:
	\begin{itemize}
	\item i) For all $c \in \mathcal{C}$, $u(c | c_\ast, c^\ast)$ exists and is unique.
	\item ii) The value of $u(c | c_\ast, c^\ast)$ is unafected by the occurrence of any event $G \succ \emptyset$.
	\item iii) $0 = u(c_\ast | c_\ast, c^\ast) \leq u(c | c_\ast, c^\ast) \leq u(c^\ast | c_\ast, c^\ast) = 1$.
	\end{itemize}
	\end{prop}
	\begin{defi}
	For a bounded decision problem and any event $G \succ \emptyset$ and $a = \{ c_j | E_j : j \in J \}$ we define the conditional expected utility of $a$ as
	\[ \bar{u}(a | c_\ast, c^\ast, G) = \sum_{j \in J} u(c_j | c_\ast, c^\ast)P(E_j | G). \]
	If $G = \Omega$ we simply denote $\bar{u}(a | c_\ast, c^\ast, G)$ as $\bar{u}(a | c_\ast, c^\ast)$
	\end{defi}
	
	\begin{teo}
	For any bounded decision problem with extreme consequences $c^\ast \succ c_\ast$ and any $G \succ \emptyset$
	\[ a_1 \succeq_G a_2 \Leftrightarrow \bar{u}(a_1) \geq \bar{u}(a_2). \]
	\end{teo}
	\begin{proof}
	See \cite{bernardo2000bayesian} Chapter 2 “Foundations”, Proposition 2.22.
	\end{proof}
	This result means that for any decision maker whose preferences satisfies axioms 1 through 5 the only election criteria that is compatible with the axioms is the maximization of expected utility and thus establishes a \textit{normative} criteria for decision making for rational agents. Of course, this result establishes a complete ordering of the options, but it does not guarantees the existence of an option for which the expected utility is maximum and further mathematical assumptions are required over the utility function $u$ in order to guarantee the existence of the maxmimum utility option (\cite{bernardo2000bayesian}).
	
	In the particular case where the algebra $\mathcal{E}$ contains only the set $\Omega$ as its only element we say that it is a decision problem without uncertainty and in this case applying the previous Theorem we obtain the decision criteria which consist in maximizing the utility function.
	
	The result extends previous work of \cite{von1944theory} who assumed that the decision maker knows the probabilties of the uncertain events and guarantees the existence of an utility function such that the decision maker chooses by calculating expected value of this utility function.
	\subsection{Game Theory}
	A \textit{game} arises when two or more \textit{rational} decision makers, or players, have to make decisions in a situation in which the outcome for each player is partly determined by the choices of other players (\cite{binmore2008rational}).
	
	Game Theory is an area of Mathematics which is used to model games. The basic assumptions of the theory is that each decision maker is rational, and they take into account their knowledge or expectations of other decision makers behavior.
	
	In this section we will review the basic models of Game Theory that are required to this thesis proposal, which are the normal-form game, and the extensive-form game. 
	\subsubsection{Building blocks}
	The normal-form game models a situation in which two or more agents interact and where each one of them will choose an action simultaneously. Also, it is assumed that all of the relevant aspects of the game are known for each player.
	
	We follow the exposition from \cite{osborne1994course}, \cite{binmore2007playing}, \cite{shoham2008multiagent}.
	
	\begin{defi}
	A normal-form game is a tuple $(N,A,(\succeq_i)_{i \in N})$ where:
	\begin{itemize}
	\item $N$ is a finite set of players.
	\item $A=A_1 \times ... \times A_n$ where each $A_i$ is the set of available actions for player $i$. Each $a=(a_1,...,a_n) \in A$ is called an action profile.
	\item For each player $i \in N$, a preference relation $\succeq_i$ defined over $A$.
	\end{itemize}
	\end{defi}

Because of the decision-making criteria shown in the last section we know that each player can replace his preferences $\succeq_i$ for a utility function $u_i$.

The prisoner's dilemma is the most common example of a normal-form game, where two suspects must either confess or remain silent about a crime and they can't communicate with each other. In this game, both of the players know the consequences of each \textit{outcome} and the rewards (utilities) associated with each outcome.

In some cases, the players do not know all of the relevant aspects of a game, such as the payoffs or the available actions to other players. It is this situation that is modeled by normal-form Bayesian games (\cite{osborne1994course}, \cite{shoham2008multiagent}).

\begin{defi}
A normal-form Bayesian game consists of:
\begin{itemize}
\item A set of states $\mathcal{S}$.
\item A finite set of players $N$.
\item For each player: a set of actions $A_i$ and as in the previous definition we define $A= A_1 \times \cdots \times A_n$.
\item For each player: a finite set $T_i$ of \textit{signals} that are observable to player $i$.
\item For each player: a signal function $\tau_i : \mathcal{S} \to T_i$.
\item For each player: a probability measure $P_i$ defined over $\mathcal{S}$ such that $P(\tau^{-1}_i (t_i)) >0$ for $t_i \in T_i$.
\item For each player: a rational preference relation $\succeq_i$ defined over the set of probability measures defined over $A \times \mathcal{S}$.
\end{itemize}
\end{defi}

\subsubsection{Repeated Games}
\subsubsection{Learning in Games}

\newpage
\section{Problem Statement}
	\subsection{Problem Statement}
	Let  $\mathcal{G}$ a causal graphical model and let $(\mathcal{A},\mathcal{E},\mathcal{C})$ a decision problem under uncertainty whose actions $a_i = \{ c_j | E_j : j \in J \}$  are causally related to consequences $c \in \mathcal{C}$ through the uncertain events $E \in \mathcal{E}$ which correspond to variables in $\mathcal{G}$. And consider a rational decision maker who doesn't know the parameters of the causal model which control the probabilities of observing a consequence given an action $a \in \mathcal{A}$, we ask how the decision maker could learn about the causal structure that controls his environment in order to make good choices with respect to the decision makers' preferences.
	\subsection{Motivation}
	Decision making under uncertain conditions is a fundamental part of intelligent reasoning (\cite{lake2017building}). Intelligent agents often face situations where an action must be chosen in the presence of uncertain conditions; this means that an outcome will be observed according to some probability distribution given the action chosen by the agent.
	
	In many real-world applications, the agent doesn't know all of the parameters required to calculate the maximum utility, but if the agent knew that his actions and the possible consequences were \textit{causally related}, then he could attempt to discover this relations and use them in order to predict consequences of actions better than if he only observes multiple action-outcome pairs as done in Reinforcement Learning (\cite{sutton1998reinforcement}). 
	
	It is known that human beings conceive their actions on the world as \textit{intervening} in the world (\cite{hagmayer2009decision}). Following this idea, \cite{lattimoreNIPS2016} consider decision problems where the action to be chosen is an intervention over a known causal graphical model. The agent must choose the intervention that maximizes the value of a \textit{target variable} after a series of learning rounds. They model their problem as if choosing an intervention was choosing an arm of a \textit{slot machine}, in which a gambler chooses an arm and gets some reward. From the rewards they estimate probabilities and output an optimal action in the sense of obtaining minimal regret. Their work considers that the causal model is fully known. They mention that the case where the causal model is unknown is left as an open question, and it is precisely what we are proposing to answer. 
	\subsection{Justification}
	Many real-world applications of decision making are solved by \textit{associative} methods which capture only statistical patterns that are found in data. For example, current methods in Reinforcement Learning, and specifically in Deep Reinforcement Learning, although they have good performance in the task that were supposed to solve, they can not explain \textit{why} a specific trajectory was chosen by the algorithm. This is highly relevant in real-world applications such as self-driving cars, where it is very important to understand why an accident happened. 
	
	For example, as told by \cite{bornstein2016artificial}, at the University of Pittsburgh Medical Center a team of researchers tried to use Machine Learning to predict whether pneumonia patients might develop severe complications. For this purpose they trained Neural Networks and Decision Trees using the hospital's own data. Neural Networks outperformed Decision Trees, but only by studying the decisions made by the latter did the doctors find out that the algorithms instructed doctors to send home pneumonia patients who already had asthma, despite the fact that asthma sufferers are known to be extremely vulnerable to complications. The problem relied in the training data, because the hospital policy was to automatically send asthma sufferers with pneumonia to intensive care, and this policy worked so well that asthma sufferers almost never developed severe complications. It was only through the interpretability of the Decision Trees that the doctors didn't send asthma patients with pneumonia home to a certain death. 
	
	Methods based in Deep Neural Networks aren't supposed to explain why a certain output was produced since those methods are based in \textit{parallel distributed representations} and the same goes for any other learning algorithm that uses Deep Neural Networks, like Deep Reinforcement learning; when AlphaGo (\cite{silver2017mastering}) defeated the world-champion, humanity didn't learn anything new about the game of Go, because the algorithm was not designed to explain its moves, it was just curve fitting.
	
	Learning a causal model of an environment and using it to act upon the environment allows to \textit{explain} aspects of the model that a purely associative model would not be abel to explain. It allows to ask \textit{why}. 
	
	
	\subsection{Research Question}
	How a rational decision maker who faces an uncertain environment which is governed by a causal mechanism can learn and make use of this causal structure in order to make good choices? How can causal structure help a decision maker in order to guide his learning process?
	\subsection{Hypothesis}
	Let $\mathcal{G}$ a causal graphical model and let $(\mathcal{A},\mathcal{E},\mathcal{C})$ a decision problem whose actions $a_i = \{ c_j | E_j : j \in J \}$  are causally related to consequences $c \in \mathcal{C}$ through the uncertain events $E \in \mathcal{E}$ which correspond to variables in $\mathcal{G}$. Then, if a decision maker doesn't know the probabilities of the uncertain events in $E$, by repeatedly making decisions he can learn causal information and use it to find the optimal actions (in the sense of expected utility theory) in less, or equal, rounds than if he doesn't consider causal information.
	\subsection{Objectives}
		\subsubsection{General Objectives}
		\subsubsection{Specific Objectives}
		\subsubsection{Limitations}

\newpage
\section{Related Work}

\newpage
\section{Methodology}
\subsection{Modelling a Decision Problem under Uncertainty as a Game}
\subsubsection{Fully knowing the causal graphical model}
\subsubsection{Knowing only the graph structure}
\subsubsection{General case: Model unknown}
\subsection{Validation of the proposed methods}
\subsection{Experimental methodology}
\subsection{Working plan}
\subsubsection{Publication plan}
\begin{itemize}
\item Gonzalez-Soto, M., Sucar, L.E., Escalante, H.J., \textbf{Playing against Nature: causal discovery for decision making under uncertainty}. Accepted for a poster presentation at the CausalML 2018 workshop at ICML 2018.
\end{itemize}

\newpage
\section{Preliminary Results}
To show the factibility of this proposal, we considered a test scenario and two cases in ascending difficulty.
\subsection{Test scenario}
Consider a sick patient who arrives at a hospital and he either has disease $A$ or disease $B$. The doctor can either give him some pill or send him into surgery.  Both treatments entail risks and whether the treatment cures the patient or not depends on which disease it had originally. The doctor could be facing a mutation from a known disease, so she has some knowledge about what could happen if a treatment is given to the patient. Using her previous knowledge as a true model, she can choose a treatment and observe the outcome from which she will learn about this disease, so she could make a better decision the next time a similar patient arrives.

The causal model that governs this situation is shown in Figure \ref{causal_model}. The parameters for this model were fixed intuitively in such a way that each treatment is effective for only one disease, but the most effective treatment is riskier.

The variables in the model are: 
\begin{itemize}
\item \textbf{Disease:} Either $A$ or $B$.
\item \textbf{Treatment:} Either pill or surgery.
\item \textbf{Reaction:} Either dying or surviving.
\item \textbf{Lives:} Either living or dying.
\end{itemize}

The variables are causally related as shown in Figure \ref{causal_model}.

\begin{figure}[ht]
\vskip 0.2in
\begin{center}
\centerline{\includegraphics[scale=0.5]{/Users/MauricioGS1/INAOE/Segundo_Semestre/Seminario/Semana12/codigo/causal_graph.png}}
\caption{Causal graphical model for the test scenario: the target variable \textit{Lives} is causally influenced by the disease the patient has, the treatment assigned and the survival to the secondary effects of treatment.}
\label{causal_model}
\end{center}
\vskip -0.2in
\end{figure}

The variable \textit{Lives} is the \textit{target variable} and, in this example, the only variable that can be intervened upon is the variable \textit{Treatment}. The decision maker prefers an outcome in which the patient lives.

In this scenario, Nature's move will consist in randomly assigning a disease to the patient. Then, the medic will asign a treatment using his current beliefs about the disease and the possible outcomes. The decision nodes for this play of the medic form an \textit{information set} because the medic doesn't know how she arrived there since she doesn't know what disease did Nature assign. Finally, Nature will sample the consequence of the treatment from the causal model and the medic will observe the outcome.

For this test scenario whose causal graphical model is shown in Figure \ref{causal_model}, we see  by applying the Pearl's do-calculus that the interventional distribution $P_{do(Tr)}(Y)$ is given by
\[ P(Y | do(Tr))=P(Y | D, Tr, R)P(R | Tr) P(D). \]
\\
In fact, from the structure of the model, which is shown in Figure \ref{causal_model}, we see that the involved probabilities in any calculations are:
\[ P(\textrm{disease}), P(\textrm{treatment}), P(\textrm{reaction} | \textrm{treatment}), \]
and
\[P(\textrm{lives} | \textrm{disease, treatment, reaction}). \]
\\
We can also see that the joint distribution for all of the variables can be expressed as
\[ P(Y | D, Tr, R)P(R | Tr) P(D)P(Tr). \]
This expression will be useful when specifying beliefs about the model as Dirichlet distributions.

\subsection{Case 1: The causal model is completely known}
If the causal model is completely known to the decision maker, then in one step she can obtain the probability for her desired value of the target variable, which in this case is the value corresponding to the outcome in which the patient lives at the end. Using this probability, she can choose which treatment to assign. Since this action maximizes the probability of the occurence desired value, it maximizes the expected utility, and it is also a \textit{best response} to the player Nature.

\subsection{Case 2: Only the structure is known}
From the expression of the joint probability we notice that we need to specify a distribution over each one of these distributions, which will be each one a Dirichlet distribution.

We begin with a random assignation of the $\alpha$ parameter for each of the distributions considered. We use Dirichlet distribution for each of the conditional probability tables that appear in the factorization of the joint probability for the graph of $\mathcal{G}$. Since each of the variables in the model is binary, then the product of these Dirichlets is Dirichlet.

With this parameters, the decision maker forms a causal model and chooses the action that maximizes the probability of the desired value for target variable as in Case 1.  With this action chosen, we simulate an outcome from the causal graphical model using the chosen action as an intervention. This evidence is used to update the parameters, which then will be used to generate a new causal model, and so on.

We show the results of two experiments. We compare the performance obtained by the causal agent, a \textit{random agent} who selects his actions at random, and an agent performing Q-learning. We show the average perform over $20, 50, 100$ and $200$ rounds.


\begin{center}
\begin{minipage}{0.48\linewidth}
\includegraphics[width=\linewidth]{/Users/MauricioGS1/INAOE/Segundo_Semestre/CausalML_Workshop/figures/20_rounds_format.png}
\captionof{figure}{Average reward in 20 rounds}
\end{minipage}%
\hfill
\begin{minipage}{0.49\linewidth}
\includegraphics[width=\linewidth]{/Users/MauricioGS1/INAOE/Segundo_Semestre/CausalML_Workshop/figures/50_rounds_format.png}
\captionof{figure}{Average reward in 50 rounds}
\end{minipage}
\end{center}
\begin{center}
\begin{minipage}{0.49\linewidth}
\includegraphics[width=\linewidth]{/Users/MauricioGS1/INAOE/Segundo_Semestre/CausalML_Workshop/figures/100_rounds_format.png}
\captionof{figure}{Average reward in 100 rounds}
\end{minipage}
\hfill
\begin{minipage}{0.49\linewidth}
\includegraphics[width=\linewidth]{/Users/MauricioGS1/INAOE/Segundo_Semestre/CausalML_Workshop/figures/200_rounds_format.png}
\captionof{figure}{Average reward in 200 rounds}
\end{minipage}
\end{center}



%In Figure \ref{20_rounds} we observe the average rewards for each agent in 20 rounds of decision making. Here we notice that Q-learning outperforms our algorithm, which has a similar performance as the random choosin procedure until round 11.
%
%\begin{figure}[ht]
%\vskip 0.2in
%\begin{center}
%\centerline{\includegraphics[scale=0.5]{/Users/MauricioGS1/INAOE/Segundo_Semestre/CausalML_Workshop/figures/20_rounds_format.png}}
%\caption{Average reward obtained in each round for each agent}
%\label{20_rounds}
%\end{center}
%\vskip -0.2in
%\end{figure}
%
%In Figure \ref{50_rounds} we observe the average rewards for each agent in 50 rounds of decision making. Our algorithm follow closely the Q-learning agent and outperform the random agent.
%
%\begin{figure}[ht]
%\vskip 0.2in
%\begin{center}
%\centerline{\includegraphics[scale=0.5]{/Users/MauricioGS1/INAOE/Segundo_Semestre/CausalML_Workshop/figures/50_rounds_format.png}}
%\caption{Average reward obtained in each round for each agent}
%\label{50_rounds}
%\end{center}
%\vskip -0.2in
%\end{figure}
%
%In Figure \ref{100_rounds} we observe the average reward obtained by the three agents in 100 rounds, where our algorithm slightly outperforms Q-Learning.
%
%\begin{figure}[ht]
%\vskip 0.2in
%\begin{center}
%\centerline{\includegraphics[scale=0.5]{/Users/MauricioGS1/INAOE/Segundo_Semestre/CausalML_Workshop/figures/100_rounds_format.png}}
%\caption{Average reward obtained in each round for each agent}
%\label{100_rounds}
%\end{center}
%\vskip -0.2in
%\end{figure}
%
%In Figure \ref{200_rounds} we observe the average reward obtained by the three agents in 200 rounds. The average reward obtained is very similar for Q-learning and our algorithm.
%
%\begin{figure}[ht]
%\vskip 0.2in
%\begin{center}
%\centerline{\includegraphics[scale=0.5]{/Users/MauricioGS1/INAOE/Segundo_Semestre/CausalML_Workshop/figures/200_rounds_format.png}}
%\caption{Average reward obtained in each round for each agent}
%\label{200_rounds}
%\end{center}
%\vskip -0.2in
%\end{figure}

We see that our method obtains a very similar reward as the classic Q-learning algorithm for a larger number of rounds, where the random agent is outperformed, but our model offers something extra because it learns a causal model of the environment. 

\newpage
\bibliographystyle{apalike}
\bibliography{/Users/MauricioGS1/INAOE/Propuesta/Bibliografia.bib}
\end{document}
